\documentclass[12pt]{article}
\usepackage{../be-my-concrete}
\usepackage[
    a4paper,
    margin = 1.5cm
]{geometry}

\pagestyle{empty}

\begin{document}
\centering
\Huge 
Геометрия-0,5 \bigskip

\LARGE
Модульный курс на У2 \bigskip

Егор Лунёв \bigskip

Окончившим 8--10 класс 

\bigskip
\bigskip
\vfill
\begin{asy}
    size(9cm);
    triangle t = triangleabc(9, 10, 11); draw(t, linewidth(bp));

    point H_a = foot(t.VA); point H_b = foot(t.VB); point H_c = foot(t.VC);
    point H = orthocentercenter(t);

    draw(segment(t.VA, H_a), grey); draw(segment(t.VB, H_b), grey); draw(segment(t.VC, H_c), grey);

    perpendicularmark(t.AB, line(t.VC, H_c), dashed + blue, size=10);
    perpendicularmark(t.BC, line(t.VA, H_a), blue, size=10, quarter=1);
    perpendicularmark(t.AC, line(t.VB, H_b), blue, size=10, quarter=3);

    draw(circle(t.VC, H_a, H_b));
    draw(H_a--H_b, grey);

    markangle(n = 1, H_a, H, point(t.VC), radius=12, red);
    markangle(n = 1, H_a, H_b, point(t.VC), radius=14, red);

    clipdraw(circle(t.VA, H_a, H_b));

    //draw(arc(circle(t.VA, H_a, H_b), -5, 185));

    markangle(n = 1, H_a, point(t.VB), H_c, radius=16, red);

    draw(circle(H, H_a, H_c), dashed+red+bp*0.7);
\end{asy}
\qquad
\begin{asy}
    size(8cm, 6cm);
    triangle t = triangleabc(11.1, 12.3, 11);
    draw(t, linewidth(bp)); 

    circle omega = circle(t.A, t.B, t.C); draw(omega);

    line b = bisector(t.VC); 
    point L = intersectionpoints(b, omega)[0]; dot(L);
    draw(t.C--L); draw(t.A--L, dashed+blue, StickIntervalMarker(1, 1, size=7)); draw(t.B--L, dashed+blue, StickIntervalMarker(1, 1, size=7));

    markangle(n = 1, t.A, t.C, L, radius = 20, red);
    markangle(n = 1, L, t.C, t.B, radius = 25, red);
\end{asy}
\qquad
\begin{asy}
    size(8cm, 6cm);
    triangle t = triangleabc(11, 9, 10); draw(t, linewidth(bp)); 

    point M = midpoint(t.AB);
    draw(t.A--t.B, StickIntervalMarker(2, 1, size=7));
    line m = line(t.C, M); draw(m);

    point Pa = projection(m)*t.A; 
    point Pb = projection(m)*t.B; 

    draw(Pa--t.A, blue+dashed, StickIntervalMarker(1, 2, size=5)); draw(Pb--t.B, blue+dashed, StickIntervalMarker(1, 2, size=5));

    perpendicularmark(line(t.VA, Pa), m, quarter = 2, red, size=7);
    perpendicularmark(line(t.VB, Pb), m, quarter = 2, red, size=7);

    dot(Pa);
    dot(Pb);
    dot(M);
    draw(box((-1, -2), (0,0)), invisible);
\end{asy}
\qquad
\begin{asy}
    size(8cm, 6cm);
    circle omega = circle(origin(), 1); draw(omega, blue);

    point A = angpoint(omega, 200);
    point B = angpoint(omega, 144); 
    point C = angpoint(omega, 80); 
    point D = angpoint(omega, -20); 

    point M = angpoint(omega, 172);
    point N = angpoint(omega, 114);
    point K = angpoint(omega, 30);
    point L = angpoint(omega, -90);

    draw(A--B--C--D--A);

    draw(A--M, gray, StickIntervalMarker(1, 1, size=5));
    draw(B--M, gray, StickIntervalMarker(1, 1, size=5));
    draw(B--N, gray, StickIntervalMarker(1, 2, size=5));
    draw(C--N, gray, StickIntervalMarker(1, 2, size=5));
    draw(C--K, gray, StickIntervalMarker(1, 3, size=5));
    draw(D--K, gray, StickIntervalMarker(1, 3, size=5));
    draw(D--L, gray, StickIntervalMarker(1, 4, size=5));
    draw(A--L, gray, StickIntervalMarker(1, 4, size=5));

    draw(N--L, dashed+red);
    draw(M--K, dashed+red);

    perpendicularmark(line(M, K), line(N, L), blue);

    dot("$A$", A, dir(200));
    dot("$B$", B, dir(145));
    dot("$C$", C, dir(80));
    dot("$D$", D, dir(-20));

    dot("$M$", M, dir(172));
    dot("$N$", N, dir(114));
    dot("$K$", K, dir(30));
    dot("$L$", L, dir(-90));
\end{asy}
\qquad
\begin{asy}
    size(8cm, 6cm);

    circle omega = circle(origin(), 1); draw(omega, blue);

    point A = angpoint(omega, 160);
    point B = angpoint(omega, 100);
    point C = angpoint(omega, 20);
    point D = angpoint(omega, -100);

    draw(A--B--C--D);
    draw(A--C, gray);
    draw(B--D, gray);

    point P = intersectionpoint(line(A,C), line(B, D)); 

    perpendicularmark(line(A, C), line(B, D), red);


    point M = (A+D)/2; draw(A--D, StickIntervalMarker(2, 1, size=7));

    draw(line(M, P), dashed+blue);

    perpendicularmark(line(M, P), line(B,C), quarter=3, red);



    dot(M);
    dot("$P$", P, dir(-55));
    dot("$A$", A, dir(160));
    dot("$B$", B, dir(100));
    dot("$C$", C, dir(20));
    dot("$D$", D, dir(-100));
\end{asy}
\qquad
\begin{asy}
    size(12cm, 6cm);
    point O = (2, 0);
    circle omega = circle(O, 1.7); draw(omega);
    point P = (-2.5, 2); dot("$P$", P, dir(70));

    point A_2 = angpoint(omega, 40); dot("$A_2$", A_2, dir(40));
    point A_1 = intersectionpoints(omega, line(P, A_2))[1]; dot("$A_1$", A_1, dir(100));
    draw(line(P, A_2));

    point B_2 = angpoint(omega, -80); dot("$B_2$", B_2, dir(-100));
    point B_1 = intersectionpoints(omega, line(P, B_2))[1]; dot("$B_1$", B_1, dir(-140));
    draw(line(P, B_2));

    draw(B_1--A_2, red+dashed); draw(B_2--A_1, red+dashed);

    markangle(n = 1, P, A_2, B_1, radius=25, blue);
    markangle(n = 1, A_1, B_2, P, radius=25, blue);

    markangle(n = 2, B_2, P, A_2, radius=20, blue+bp);

    draw(box((-3,-2), (4, 2.3)), invisible);
\end{asy}
\vfill
\end{document}
