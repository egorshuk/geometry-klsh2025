\begin{tasks}
    \item Докажите, что точка, симметричная точке пересечения высот (ортоцентру) треугольника относительно стороны, лежит на описанной окружности этого треугольника.

    \item Пусть точка $O$ --- центр описанной окружности треугольника $ABC$, $AH$ --- высота. Докажите, что $\angle BAH = \angle OAC$.
        
    \item Пусть $AA_1$ и $BB_1$ --- высоты остроугольного треугльника $ABC$, а точка $O$ --- центр его описанной окружности. Докажите, что $CO \perp A_1B_1$.

    \item В треугольнике $ABC$ проведены высоты $BB_1$ и $CC_1$, а также отмечена точка $M$ --- середина стороны $BC$. Точка $H$ --- его ортоцентр, а точка $P$ --- пересечения луча \texttt{(!)} $MH$ с окружностью $(ABC)$. Докажите, что точки $P, A, B_1, C_1$  концикличны. 

    \item Во вписанном четырехугольнике $ABCD$ точка $P$ --- точка пересечения диагоналей $AC$ и $BD$. Точка $O$ --- центр окружности $(ABP)$. Докажите, что $OP \perp CD$. 

    \item \named{Муниципальный этап ВСОШ (Москва), 2020, 9.4}{Пусть точки $B$ и $C$ лежат на по\-лу\-окруж\-но\-сти с диаметром $AD$. Точка $M$ --- середина отрезка $BC$. Точка $N$ такова, что точка $M$ --- середина отрезка $AN$, докажите что $BC \perp ND$}. 

    \item В треугольнике $ABC$ проведена высота $AD$ и отмечен центр описанной окружности --- $O$. Пусть точки $E$ и $F$ --- проекции точек $B$ и $C$ на прямую $AO$. $N$ --- точка пересечения прямых $AC$ и $DE$, а $M$ --- точка пересечения прямых $AB$ и $DF$. Докажите, что точки $A, D, N, M$ концикличны.

    \item Окружность $S_2$ проходит через центр $O$ окружности $S_1$ и пересекает ее в точках $A$ и $B$. Через точку A проведена касательная к окружности $S_2$; $D$ --- вторая точка пересечения этой касательной с окружностью $S_1$. Докажите, что $AD = AB$.

    \item \named{Baltic Way, 2019, problem 12}{Let $ABC$ be a triangle and $H$ its orthocenter. Let $D$ be a point lying on the segment $AC$ and let $E$ be the point on the line $BC$ such that $BC \perp DE$. Prove that $EH \perp BD$ if and only if $BD$ bisects $AE$}. 

    \item \named{Лемма Архимеда}{Две окружности касаются внутренним образом в точке $M$. Пусть $AB$ --- хорда большей окружности, касающаяся меньшей окружности в точке $T$. Докажите, что $MT$ ---  биссектриса угла $AMB$.}

    \item В трапеции $ABCD$ с основаниями $AB$ $CD$ выполнено равенство  $AB = BD+CD$. Пусть $𝐸$ --- середина $𝐴𝐶$. Докажите, что $\angle BED = 90^\circ$.

    \item В параллелограмме $ABCD$ диагональ $AC$ больше диагонали $BD$. Точка $M$ на диагонали $AC$ такова, что около четырехугольника $BCDM$ можно описать окружность. Докажите, что $BD$ --- общая касательная окружностей, описанных около треугольников $ABM$ и $ADM$.

    \item \named{Прямая Симсона}{Докажите, что основания перпендикуляров, опущенных из произвольной точки описанной окружности на стороны треугольника (или их продолжения), лежат на одной прямой.}

    \item Пусть $H$ --- ортоцентр остроугольного треугольника $ABC$ Серединный перпендикуляр $\ell$ к стороне $AC$ пересекает прямые $AH$, $CH$ в точках $K$ и $L$ соответственно. Докажите, что ортоцентр треугольника  лежит на прямой, содержащей одну из медиан треугольника $ABC$.

    \end{tasks}
