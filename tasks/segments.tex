\begin{tasks}
    \item Диагонали $AC$ и $BD$ вписанного в окружность четырехугольника $ABCD$ взаимно перпендикулярны и пересекаются в точке $M$. Известно, что $AM = 3$, $BM = 4$ и $CM = 6$. Найдите $CD$.
    \item Через точку $M$ проведены две прямые. Одна из них касается некоторой окружности в точке $A$, а вторая пересекает эту окружность в точках $B$ и $C$, причем $BC = 7$ и $BM = 9$. Найдите $AM$.
    \item Дана точка $P$, удаленная на расстояние, равное $7$, от центра окружности, радиус которой равен $11$. Через точку $P$ проведена хорда, равная $18$. Найдите отрезки, на которые делится хорда точкой $P$.
    \item Точка $M$ лежит внутри окружности радиуса $R$ и удалена от центра на расстояние $d$. Докажите, что для любой хорды $AB$ этой окружности, проходящей через точку $M$, произведение $AM \cdot BM$ одно и то же. Чему оно равно?
    \item В квадрат $ABCD$ со стороной $a$ вписана окружность,
которая касается стороны $CD$ в точке $E$. Найдите хорду, соединяющую точки, в которых окружность пересекается с прямой $AE$.
    \item Диагональ $AC$ вписанного в окружность четырехугольника $ABCD$ является биссектрисой угла $BAD$. Докажите, что прямая $BD$ отсекает от треугольника $ABC$ подобный ему треугольник.
    \item Две окружности пересекаются в точках $A$ и $B$. Проведены хорды $AC$ и $AD$ этих окружностей так, что хорда одной окружности касается другой окружности. Найдите $AB$, если $CB = a$, $DB = b$.
    \item Докажите, что прямая, проходящая через точки пересечения двух окружностей, делит пополам общую касательную к ним.
    \item Продолжение медианы треугольника $ABC$, проведенной из вершины $A$, пересекает описанную окружность в точке $D$. Найдите $BC$, если $AC = DC = 1$.
    \item Сторона $AD$ квадрата $ABCD$ равна $1$ и является хордой некоторой окружности, причем остальные стороны квадрата лежат вне этой окружности. Касательная $BK$, проведенная из вершины $B$ к этой же окружности, равна $2$. Найдите диаметр окружности.
    \item Точка $B$ расположена между точками $A$ и $C$. На отрезках $AB$ и $AC$ как на диаметрах построены окружности. Прямая, перпендикулярная $AC$ и проходящая через точку $B$, пересекает большую окружность в точке $D$. Прямая, проходящая через
точку $C$, касается меньшей окружности в точке $K$. Докажите, что $CD = CK$.
    \item Постройте окружность, проходящую через две данные точки и касающуюся данной прямой.
    \item Докажите, что квадрат биссектрисы треугольника равен произведению сторон, ее заключающих, без произведения отрезков третьей стороны, на которые она разделена биссектрисой.
\end{tasks}
