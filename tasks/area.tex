% TODO: написать решения

\begin{tasks}
    \item\label{task:easy-varinion} Площадь прямоугольника равна 24. Найдите площадь четырехугольника с вершинами в серединах сторон прямоугольника.
    \item Средняя линия треугольника разбивает его на треугольник и четырехугольник. Какую часть составляет площадь полученного треугольника от площади исходного?
    \item Точка $M$ расположена на стороне $BC$ параллелограмма $ABCD$. Докажите, что площадь треугольника $AMD$ равна половине площади параллелограмма.
    \item Пусть $M$ --- точка на стороне $AB$ треугольника $ABC$, причем $AM : MB = m : n$. Докажите, что площадь треугольника $CAM$ относится к площади треугольника $CBM$ как $m : n$.
    \item Точки $M$ и $N$ --- соотвественно середины противоположных сторон $AB$ и $CD$ параллелограмма $ABCD$, площадь которого равна 1. Найдите площадь четырехугольника, образованного пересечениями прымях $AN$, $BN$, $CM$, $DM$.
    \item На сторонах $AB$ и $AC$ треугольника $ABC$, площадь которого равна 50, взяты соответственно точки $M$ и $K$ так, что $AM : MB = 1 : 5$, а $AK : KC = 3 : 2$. Найдите площадь треугольника $AMK$.
    \item Прямая, проведенная через вершину $C$ трапеции $ABCD$ параллельно диагонали $BD$, пересекает продолжение основания $AD$ в точке $M$. Докажите, что треугольник $ACM$ равновелик трапеции $ABCD$.
    \item Докажите, что медианы треугольника делят его на шесть равновеликих частей.
    \item Медианы $BM$ и $CN$ треугольника $ABC$ пересекаются в точке $K$. Докажите, что четырехугольник $AMKN$ равновелик треугольнику $BKC$.
    \item Точка внутри параллелограмма соединена со всеми его вершинами. Докажите, что суммы площадей треугольников, прилежащих к противоположным сторонам параллелограмма, равны между собой.
    \item Середины сторон выпуклого четырехугольника последовательно соединены отрезками. Докажите, что площадь полученного четырехугольника вдвое меньше площади исходного.\footnote{Привет \cref{task:easy-varinion}!}
    \item Отрезки, соединяющие середины противоположных сторон выпуклого четырехугольника, взаимно перпендикулярны и равны 2 и 7. Найдите площадь четырехугольника.
        \moditem{*} Докажите, что сумма расстояний от произвольной точки внутри равностороннего треугольника до его сторон всегда одна и та же.
    \item Докажите, что площадь треугольника равна произведению полупериметра треугольника и радиуса вписанной окружности.
    \item Дан треугольник $ABC$. Найдите геометрическое место таких точек $M$, для которых:
        \begin{tasks}
        \item треугольники $AMB$ и $ABC$ равновелики;
        \item треугольники $AMB$ и $AMC$ равновелики;
        \item треугольники $AMB$, $AMC$ и $BMC$ равновелики.
        \end{tasks}
    \item Окружность касается стороны треугольника, равной $a$, и продолжения двух других сторон. Докажите, что радиус окружности равен площади треугольника, деленной на разность между полупериметром и стороной $a$.
    \item Боковая сторона $AB$ и основание $BC$ трапеции $ABCD$ вдвое меньше ее основания $AD$. Найдите площадь трапеции, если $AC = a$, $CD = b$.
    \item Из середины каждой стороны остроугольного треугольника опущены перпендикуляры на две другие стороны. Докажите, что площадь ограниченного ими шестиугольника равна половине площади треугольника.

        % тут если будет слишко легко --- дать просто задачи 3 уровня из гордина. 
\end{tasks}
