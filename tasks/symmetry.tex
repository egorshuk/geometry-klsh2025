\begin{tasks}
    \item Пусть $M$ и $N$  --- середины оснований трапеции. Докажите, что если прямая $MN$ перпендикулярна основаниям, то трапеция равнобедренная.
    \item Пусть $M$ --- середина отрезка $AB$. Точки $A'$, $B'$ и $M'$ --- образы точек соответственно $A$, $B$ и $M$ при симметрии относительно некоторой точки $O$. Докажите, что $M'$ --- середина $A'B'$.
    \item На противоположных сторонах параллелограмма как на сторонах построены вне параллелограмма два квадрата. Докажите, что прямая, соединяющая их центры, проходит через центр параллелограмма.
    \item Докажите, что точки, симметричные произвольной точке относительно середин сторон квадрата, являются вершинами некоторого квадрата.
    \item Даны две концентрические окружности $S_1$ и $S_2$. Постройте прямую, на которой эти окружности высекают три равных отрезка.
    \item Противоположные стороны выпуклого шестиугольника попарно равны и параллельны. Докажите, что он имеет центр симметрии.
    \item Диагонали $AC$ и $BD$ параллелограмма $ABCD$ пересекаются в точке $O$. Докажите, что окружности, описанные около треугольников $AOB$ и $COD$, касаются
    \item Фигура имеет две перпендикулярные оси симметрии. Докажите, что она имеет центр симметрии.
    \item Точки $A$ и $B$ лежат по разные стороны от прямой $\ell$. Постройте на этой прямой точку $M$ так, чтобы прямая $\ell$ делила угол $AMB$ пополам.
    \item Внутри острого угла даны точки $M$ и $N$. Как из точки $M$ направить луч света, чтобы он, отразившись последовательно от сторон угла, попал в точку $N$?

\end{tasks}
