\begin{tasks}
    \item Докажите, что высота прямоугольного треугольника, проведенная из вершины прямого угла, делит треугольник на два подобных треугольника.
    \item Точка $M$ --- середина стороны $BC$ параллелограмма $ABCD$. Найдите отношение, в котором отрезок $AM$ делит диагональ $BD$.
    \item В треугольнике $ABC$ точка $K$ на медиане $AM$ расположена так, что $AK : KM = 1 : 3$. Найдите отношение, в котором прямая, проходящая через точку $K$ параллельно стороне $AC$, делит сторону $BC$.
    \item Докажите, что медиана $AM$ треугольника $ABC$ делит пополам любой отрезок с концами на $AB$ и $AC$, параллельный стороне $BC$.
    \item \named{Замечательное свойство трапеции}{Докажите, что точка пересечения диагоналей, точка пересечения продолжений боковых сторон и середины оснований любой трапеции лежат на одной прямой.}
    \item Через точку пересечения диагоналей трапеции с основаниями a и b проведена прямая, параллельная основаниям. Найдите отрезок этой прямой, заключенный между боковыми сторонами трапеции.
    \item $AA_1$ и $BB_1$ --- высоты остроугольного треугольника $ABC$. Докажите, что треугольник $AA_1C$ подобен треугольнику $BB_1C$, а треугольник $ABC$ подобен треугольнику $A_1B_1C$.
    \item В треугольнике $ABC$ проведены высоты $BB_1$ и $CC_1$. Найдите $B_1C_1$, если $\angle A = 60^\circ$ и $BC = 6$.
    \item Дан треугольник $ABC$. На продолжении стороны $AC$ за точку $C$ взята точка $N$ так, что $CN = AC$. Точка $K$  середина стороны $AB$. В каком отношении прямая $KN$ делит сторону $BC$?
    \item Дан треугольник $ABC$. На продолжении стороны $AC$ за точку $C$ взята точка $N$ так, что $CN = 3AC$. Точка $K$ лежит на стороне $AB$, причем $AK : KB = 1 : 3$. В каком отношении прямая $KN$ делит сторону $BC$?
    \item Точки $M$ и $N$ лежат на сторонах соответственно $AB$ и $AD$ параллелограмма $ABCD$, причем $AM : MB = 1 : 2$, $AN : ND = 3 : 2$. Отрезки $DM$ и $CN$ пересекаются в точке $K$. Найдите отношения $DK : KM$, $CK : KN$.
    \item Точки $K$ и $E$ лежат соответственно на сторонах $BC$ и $AB$ треугольника $ABC$. Отрезки $AK$ и $CE$ пересекаются в точке $M$. В каком отношении прямая $BM$ делит сторону $AC$, если $BK : KC = 1 : 2$, $AE : EB = 2 : 3$?
    \item Докажите, что биссектриса треугольника делит его сторону на отрезки, пропорциональные двум другим сторонам.
    \item В треугольнике $ABC$ медиана $AD$ и биссектриса $BE$ перпендикулярны и пересекаются в точке $F$. Известно, что площадь треугольника $DEF$ равна $5$. Найдите площадь треугольника $ABC$.
    \item На сторонах $AB$, $AC$ и $BC$ правильного треугольника $ABC$ расположены точки соответственно $C_1$, $B_1$ и $A_1$, причем треугольник $A_1B_1C_1$ является правильным. Отрезок $BB_1$ пересекает сторону $C_1A_1$ в точке $O$, причем $\rfrac{BO}{OB_1} = k$. Найдите отношение площади треугольника $ABC$ к площади треугольника $A_1B_1C_1$.
\end{tasks}
