\begin{enumerate}[resume*]

    \item \task{ММО, 1994}{В треугольнике $ABC$ точки $M$ и $N$ -- проекции вершины $B$ биссектрисы углов $A$ и $C$, а $P$ и $Q$ -- проекции на внешние биссектрисы этих же углов.
    \begin{enumerate}
        \item Докажите, что точки $M$, $N$, $P$ и $Q$ коллинеарны. 

            \solution{
                По \cref{lem:255} все точки $M, N, P, Q$ лежат на средней линии треугольника $ABC$, параллельной стороне $AC$.
            }

        \item Докажите, что длина отрезка $PQ$ равна полупериметру треугольника $ABC$.

            \solution{
                Пусть $M_a, M_c$ -- середины отрезков $AB, BC$ соответствено, тогда $M_aM_c = \frac{AC}{2}$.

                Т.к. $APB, BQC$ -- прямоугольные, то $M_aQ = \frac{BC}{2}, M_cP = \frac{AC}{2}$. По \cref{lem:255} $P, Q \in M_aM_c$, значит $PQ = PM_c + M_aM_c + M_aQ = \frac{AB+AC+BC}{2}$.
            }

    \end{enumerate}}

    \item В трапецию $ABCD$ вписана окружность с центром $I$. Окружность вписанная в треугольник $ACD$ касается сторон $AD$ и $AC$ в точках $E$ и $F$. Докажите, что точки $E$, $F$ и $I$ коллинеарны. 
        
        \solution{
            Т.к. $\angle BCD + \angle ADC = 180^\circ$, то $\angle CID = 90^\circ$. Значит по \cref{lem:255} $I \in FE$.
        }
    
    \item \task{Ф. Л. Бахарев, Санкт-Петербургская о\-лим\-пи\-а\-да, 1999}{В неравнобедренном треугольнике $ABC$ проведены биссектрисы $AA_1$ и $CC_1$ и отмечены точки $K$ и $L$ -- се\-ре\-ди\-ны сторон $AB$ и $BC$ соответственно. $AP$ и $CQ$ -- перпендикуляры, опущенные на $CC_1$ и $AA_1$ соответственно. Докажите, что прямые $PK$ и $QL$ пересекаются на стороне $AC$.} 

        \solution{
            Пусть $M$ -- середина сторны $AC$. Тогда по \cref{lem:255} $P \in KM, Q \in LM$.
        }

    \item 
        \begin{enumerate}
            \item\label{lem:ex255 complex}\task{Первая внешняя Лемма 255}{Пусть $M$ и $N$ -- точки касания вневписанной окружности $\omega_a$ треугольника $ABC$ со стороной $BC$ и продолжением стороны $AC$, а $P$ -- точка пересечения бис\-сек\-три\-сы угла $A$ c прямой $MN$. Докажите, что $\angle APB = 90^\circ$.} 

                \solution{
                    \begin{theorem}\label{th:angle_ex_bisector}
                        Если в треугольника $ABC$, точка $I_a$ -- $A$-эксцентр, тогда \[
                        \angle BI_aC = 90^\circ - \frac{\angle BAC}{2}
                        .\] 
                    \end{theorem}

                    Пусть точка $P$ -- основание перпендикуляра из $I_a$ на $BP$. А точка $Q$ -- пересечение $BP$ c $AC$.
                    
                    Тогда по \cref{th:simson's line} $P, N, M$ коллинеарны, если $BQCI_a$ -- вписанный.

                    По \cref{th:angle_ex_bisector} $\angle BI_aC = 90^\circ - \frac{\angle BAC}{2}$. Но и $\angle BQA = 90^\circ - \frac{\angle BAC}{2}$.
                    Тогда по обратной \cref{lem:concycle} $BQCI_a$ -- вписанный.
                }

            \item\label{lem:ex255 simple} \task{Вторая внешняя Лемма 255}{Пусть $M$ и $N$ -- точки касания вневписанной окружности $\omega_a$ треугольника $ABC$ с продолжениями сторон $AB$ и $AC$, а $P$ -- точка пересечения бис\-сек\-три\-сы внешнего угла $B$ c прямой $MN$. Докажите, что $\angle BPC = 90^\circ$.} 

                \solution{
                    Пусть точка $P$ -- основание перпендикуляра из $C$ на $BI_a$. А точка $Q$ -- пересечение $CP$ с $AB$.

                    Тогда по \cref{th:simson's line} $P, N, M$ коллинеарны, если $AQI_aC$ -- вписанный.

                    Заметим, что $BP$ -- серединный перпендикуляр к отрезку $QC$, тогда $I_aQ = I_aC$.
                    Тогда по обратной \cref{lem:trillium} $AQI_aC$ -- вписанный.
                }

        \end{enumerate}

    \item В равнобедренном треугольнике $ABC$ $(AB = BC)$ средняя линия, параллельная стороне $BC$ пересекается со вписанной окружностью в точке $D$, не лежащей на $AC$. Докажите, что касательная к окружности в точке $D$ пересекается с биссектрисой угла $C$ на стороне $AB$. 

        \solution{
            Пусть точки $M, N$ -- середины сторон $AC, AB$ соответствено. Точка $I$ -- центр вписанной окружности. Точка $E$ -- пересечение $MN$ и $IC$. Точки $K, L$ -- точки касания вписанной окружности со сторонами $ AB, BC$ соответственно, точки $P, Q$ -- точки пересечения касательной из точки $D$ с этими же сторонами. 
            
            По \cref{lem:255} Точка $E$ лежит на $KL \parallel AC$, значит $P, E, I$ коллинеарны.
        }

    \item В треугольнике $ABC$ точки $A_c$, $B_c$, $C_c$ -- точки касания прямых $BC$, $AC$ и $AB$ с вневписанной окружностью $\omega_c$ (с центром в $I_c$). Точки $A_b$, $B_b$, $C_b$ определяются аналогично.
       
    \begin{equation*}
        \begin{cases}
            B_1 &= A_cC_c \cap A_bC_b \\
            C_1 &= A_bB_b \cap A_cB_c \\
            A_1 &= A_bB_b \cap A_cC_c \\
            A_2 &= A_cB_c \cap A_bC_b
        \end{cases}
    \end{equation*}

    \begin{enumerate}
        \item Докажите, что точки $A$, $B_1$, $C_1$, $I_b$, $I_c$ коллинеарны.

            \solution{
                Докажем для точки $B_1$. Пусть точка $X$ -- пересечение $A_cC_c$ с $AI_b$. По \cref{lem:ex255 complex} $AXC = 90^\circ$. По обратной \cref{lem:ex255 simple} $\angle AXB = 90^\circ \implies X \in A_bC_b \implies X \equiv B_1$.
                Аналогично для точки $C_1$.
            }

        \item Докажите, что \(A_1A_2 \perp BC\).

            \solution{
                Докажем, что $\angle A_cB_1A_b$ -- прямой.
                $AC_bB_1B_b$ -- дельтоид, тогда $\angle C_bB_bB_1 = \angle B_bC_bB_1 = \angle AB_1A_c = \alpha$. 

                По \cref{lem:ex255 simple} $\angle AB_1C = 90^\circ \implies CB_1\perp AB_1 \implies C_bB_b \parallel CB_1 \implies \implies CB_1A_b = \alpha.$
                Тогда $\angle A_cB_1A_b = \angle AB_1C - \angle AB_1A_c + \angle CB_1A_b = \angle AB_1C - \alpha + \alpha = 90^\circ$.

                Аналогично $A_bC_1A_c$ -- прямой. Тогда в треугольнике $A_bA_1A_c$ точка $A_2$ -- ортоцентр. Отсюда и следует, что $A_1A_2 \perp A_bA_c \equiv BC$.
            }
    \end{enumerate}
\end{enumerate}
