\begin{enumerate}[resume*]
    \item Дан прямоугольник $ABCD$. Через точку $B$ провели две перпендикулярные прямые. Первая прямая пересекает сторону $AD$ в точке $K$, а вторая — продолжение стороны $CD$ в точке $L$. $F$ -- точка пересечения $KL$ и $AC$. Докажите, что $BF \perp KL$. 

        \solution{
            Точка $B$ лежит на окружности $(DKL)$, а также $BA \perp KD$ и $BC \perp DL$, значит $AC$ -- прямая Симсона точки $B$ относительно треугольника $DKL$.

            По \cref{th:simson's line} $BF \perp KL$.
        }

    \item Пусть $AA_1$, $BB_1$, $CC_1$ -- высоты остроугольного треугольника $ABC$. Докажите, что проекции точки $A_1$ на прямые $AB$, $AC$, $BB_1$, $CC_1$ коллинеарны. 

        \solution{
            Заметим, что точка $A_1$ лежит на окружностях $(ABB_1)$ и $(ACC_1)$.

            Тогда по \cref{th:simson's line} проекции точки $A_1$ на прямые $AB, AC, BB_1$ коллинеарны. И проекции точки $A_1$ на прямые $AB, AC, CC_1$ коллинеарны. Т.к. проекции точки $A_1$ на прямые $AB, AC$ составляют прямую, и причем только одну, то все эти точки лежат на этой прямой.
        }

    \item \task{Обобщённая прямая Симсона}{$P$ -- произвольная точка описанной окружности треугольника $ABC$. Докажите, что точки $A_1$, $B_1$, $C_1$ на прямых $AC$, $BC$, $AB$ коллинеарны, когда выполняется равенство: $$\angle(AB, PC_1) = \angle(BC, PA_1) = \angle(AC, PB_1).$$}

        \solution{
            Отметим точки $B_1$ и $C_1$ правильно. Пусть прямая $B_1C_1$ пересекает $BC$ в точке $D$.

            По обратной \cref{lem:concycle} $AC_1PB_1$ -- вписанный. Тогда по \cref{th:miquel's point} все описанные окружности треугольников $ABC, AB_1C_1, A_1BC_1, A_1B_1C$ пересекаются в одной точке. Но, позвольте, $(ABC) \cap (AB_1C_1) = P$. Значит, что и $DC_1BP$, и $DCB_1P$ тоже вписанные. 

            Тогда по \cref{lem:concycle} $\angle (BC, PD) = \angle (AC, PB_1)$, а значит $D \equiv A_1$.
        }

    \item Вписанная в треугольник $ABC$ окружность касается сторон $AB$, $BC$, $CA$ в точках $C_1$, $B_1$, $A_1$ соответственно. Пусть прямая $C_1I$ пересекает прямую $A_1B_1$ в точке $P$. Тогда прямая $CP$ содержит медиану треугольника $ABC$. 

        \solution{
            Проведем через точку $P$ прямую параллельную $AB$, которая пересекает сторону $AC$ в точке $N$, а сторону $BC$ в точке $M$. 

            Заметим, что 
            \begin{equation}
                \begin{cases}
                    IP &\perp NM \\
                    IB_1 &\perp NC \\
                    IC_1 &\perp CM.
                \end{cases}
            \end{equation}

            Получается, что по \cref{th:simson's line} т.к. $A_1,B_1,P$ коллинеарны, то $I \in (CNM)$.

            Вспомним, что $CI$ -- биссектриса угла $NCM$. Тогда по \cref{lem:trillium} $I$ лежит на серединном перпендикуляре к отрезку $NM$, значит $PN = PM$.

            Сделаем гомотетию $\mathcal H_C$, чтобы $N \to A, M \to B$, точка $P$ перейдет в середину $BC$. А по \cref{def:homothety} точки $C, P$  и середина $BC$ будут лежать на одной прямой.
        }

    \item \begin{enumerate}
            \item\label{lem:simson parallel chord} Хорда $PQ$ описанной окружности треугольника $ABC$ и сторона $BC$ перпендикулярны. Докажите, что прямая Симсона точки $P$ относительно треугольника $ABC$ параллельна прямой $AQ$. 

            \solution{
                Пусть точка $B_1$ -- проекция точки $P$ на $AC$. По \cref{th:simson's line} $A_1B_1$ -- прямая Симсона точки $P$

                $PB_1A_1C$ и $PAQC$ -- вписанные. Тогда
                \begin{equation}
                    \angle CPQ = \underbrace{\angle A_1B_1C}_{\angle (A_1B_1, AC)} = \underbrace{\angle QAC}_{\angle (AQ, AC)} \implies AQ \parallel A_1B_1
                .\end{equation} 
            }

        \item \label{th:simson middle HP}\task{Закл. этап ВСОШ, 2009–2010 гг., 10.6}{Пусть $H$ -- ортоцентр треугольника $ABC$. Точки $X$ и $Y$ -- проекции точки $P$, лежащей на описанной окружности треугольника $ABC$ на стороны $AB$ и $BC$. Докажите, что середина отрезка $HP$ и точки $X$ и $Y$  коллинеарны.\footnote{Подсказка в том, что эта задача -- пункт (b). Ну и симметрии ортоцентра.}}

            \solution{
                Пусть $Z$ -- точка пересечения $XY$ с $BC$, тогда по \cref{th:simson's line} $PZ\perp BC$. Пусть точка $Q$ -- пересечение окружности $(ABC)$ c $PZ$. Тогда по \cref{lem:simson parallel chord} $AQ \parallel XY$.

                Пусть точка $H'$ симметрична $H$ относительно $BC$, точка $P'$ определяется аналогично.
                Тогда по \cref{th:side reflect} $H' \in (ABC)$, значит  $AQPH'$ и $HH'PP'$ -- равнобокие трапеции. 
                \begin{equation}
                    \angle QAH' = \angle AH'P = \angle H'HP' \implies HP' \parallel AQ \parallel XY
                .\end{equation} 

                Заметим, что $XY$ содержит в себе среднюю линию треугольника $P'HP$, а значит делит отрезок $HP$ пополам.
            }

    \end{enumerate}
    
    \item \task{Прямая Штейнера}{Пусть $P$ -- произвольная точка на описанной окружности треугольника $ABC$. Точки $P_a$, $P_b$, $P_c$ -- симметричны $P$ относительно прямых $BC$, $AC$ и $AB$ соответственно. Докажите что, точки $P_a$, $P_b$, $P_c$, $H$ коллинеарны.} 

        \solution{
            Пусть $\ell_P$ -- прямая Симсона точки $P$ относительно $ABC$. Тогда прямая содержащая точки $P_a, P_b, P_c$ получается при гомотетии $\mathcal H_P^2$.

            А по \cref{th:simson middle HP} т.к. $\ell_P$ делит отрезок $HP$ пополам, то при гомотетии \emph{прямая Штейнера} точки $P$ пройдет через $H$.
        }

    \item Пусть $\ell$ -- прямая Штейнера точки $R$ на описанной окружности $ABC$. Докажите, что если прямую $\ell$ отразить относительно стороны треугольника $ABC$, то полученная прямая пройдет через точку $R$.

        \solution{
        \begin{lemma}\label{lem:diff Shtainer}
                Пусть точка $R$ лежит на описанной окружности треугольника $ABC$. А точки $R_a, R_b, R_c$ симметричны точке $R$ относительно прямых $BC,AC, AB$. Тогда $R_a, R_b,R_c$ лежат на \emph{прямой Штейнра} точки $R$.
            \end{lemma}
        }

    \item \task{Л. А. Попов, Ф. Л. Бахарев}{Точки $A_1$, $B_1$, $C_1$ — основания высот остроугольного треугольника $ABC$ из точек $A$, $B$, $C$ соответственно. Точки $A_1$, $B_1$, $C_1$ отразили относительно средних линий треугольника, параллельных $AB$, $BC$, $CA$ соответственно, — получились точки $A_2$, $B_2$, $C_2$ соответственно. Докажите, что прямые $AA_2$, $BB_2$, $CC_2$ пересекаются в одной точке.}

        \solution{
            Пусть $M_a, M_b, M_c$ -- середины сторон $BC, AC, AB$ треугольника $ABC$.

            По \cref{def:euler's circle} точки $M_a, M_{b}, M_c, A_1, B_1, C_1$ лежат на одной окружности.

            Заметим, что точка $A$ -- отражение точки $A_1$ относительно $M_bM_c$, и точка $A_2$ отражение точки $A_1$ относительно $M_aM_b$. Значит по \cref{lem:diff Shtainer} $AA_2$ -- прямая Штейнера точки $A_1$ относительно треугольника $M_aM_bM_c$. 

            Аналогично $BB_2, CC_2$ -- тоже прямые Штейнера относительно треугольника $M_aM_bM_c$, а значит они все проходят через его ортоцентр, который также является центром описанной окружности треугольника $ABC$.
        }

    \item \task{Олимпиада им. И.Ф. Шарыгина, 2021, 8-9.6, устный тур}{\\В треугольнике $ABC$, точка $M$ -- середина стороны $BC$, точка $H$ -- ортоцентр. Биссектриса угла $A$ пересекает отрезок $HM$ в точке $T$. Окружность построенная на отрезке $AT$, как на диаметре, пересекает стороны $AB$ и $AC$ в точках $X$ и $Y$. Докажите, что точки $X$, $Y$ и $H$  коллинеарны.}

        \solution{
            Проведем прямую перпендикулярную $TH$ через точку $H$, она пересекает сторону $AB$ в точке $P$, а сторону $AC$ в точке $Q$. Тогда по \cref{th:simson's line} точки $X, Y, H$ коллинеарны, если $T \in (APQ)$. 

            Т.к. $AT$ -- биссектриса, то по \cref{lem:trillium} $T$ должна лежать на серединном перпендикуляре к отрезку $PQ$, что равносильно тому, что $M$ лежит на этом серединном перпендикуляре. Это и будем доказывать.

            Отразим точку $H$ относительно $M$, получим точку $H'$. По \cref{th:middle reflect} точка $H' \in (ABC)$, а по \cref{cor:diametr} $\angle ABH' = \angle ACH' = 90^\circ$.
            Тогда четырехугольники $HPBH'$ и $HQCH'$ -- вписанные.

            \begin{equation}
                \left\{\begin{aligned}
                        \underbrace{\angle PBH}_{90^{\circ} - \angle BAC} &= \angle HH'B \\
                        \underbrace{\angle QCH}_{90^{\circ} - \angle BAC} &= \angle HH'C
                    \end{aligned}\right| \implies \angle HH'B = \angle HH'C
            .\end{equation} 

            Значит $HH'$ является осью симметри в треугольнике $PQH'$, отсюда следует, что $PH=HQ$.
        }

    \item \task{ММО, 2006, 10.6} Точки $P$ и $Q$ лежат на описанной окружности треугольника $ABC$. На прямой $AB$ выбрана точка $C_1$ так, что $\angle(AB, PC_1) = \angle(QC_1, AB)$. Аналогично выбраны точки $B_1$ и $C_1$ на прямых $AC$ и $BC$ соответственно. Докажите, что точки $A_1$, $B_1$, $C_1$ коллинеарны.

        \solution{
            Отметим точки $P_a, P_b, P_c$ и $Q_a, Q_b, Q_c$ как симметричные точкам $P$ и $Q$, относительно прямых $BC, AC$ и $AB$ соответственно.

            Тогда по \cref{lem:diff Shtainer} данные тройки точек коллинеарны, пусть точки "семейства" $P$ лежат на прямой $Sh_p$, а точки "семейства" $Q$ лежат на $Sh_q$.
            Значит можно по-новому определить точки $A_1, B_1, C_1$, а именно 

            \begin{equation}
                \begin{cases}
                    A_1 &= P_aQ \cap Q_aP \\
                    B_1 &= P_bQ \cap Q_bP \\
                    C_1 &= P_cQ \cap Q_cP
                \end{cases}
            \end{equation}
            
            Проведем через точку $P$ прямую параллельную $Sh_q$, которая пересечет прямую $Sh_p$ в точке $X$. Аналогично определим точку $Y$.

            Тогда стороны треугольника $PPaX$ и стороны треугольника $QQ_aY$ коллинеарны. Значит по \cref{def:homothety} существует гомотетия $\mathcal H$ с отрицательным кооэфициентом, что треугольник $PP_aX$ переходит в $QQ_aY$, а значит $A_1 \in XY$. Аналогично доказывается для точек $B_1, C_1$.
        }

    \item \task{Теорема Дроз-Фарни}{Обозначим точкой $H$ -- ортоцентр треугольника $ABC$. Прямые $\ell$ и $t$ проходят через $H$ и $\ell \perp t$. Пусть $L_a$, $L_b$, $L_c$ пересечение $\ell$ с прямыми $BC$, $AC$ и $AB$ соответственно, точки $T_a$, $T_b$ и $T_c$ определяются аналогично. Докажите, что середины отрезков $T_aL_a$, $T_bL_b$, $T_cL_c$ коллинеарны.}

        \solution{
            Пусть точки $M_a, M_b, M_c$ -- середины отрезков $L_aT_a, L_bT_b, L_cT_c$ соответственно.

            Отразим $H$ относительно сторон $AB, AC, BC$, получим точки $H_c, H_b, H_a$ соответственно. Все они по \cref{th:side reflect} лежат на описанной окружности $(ABC)$.

            Построим на отрезках $L_aT_a, L_bT_b, L_cT_c$ окружности $\omega_a, \omega_b, \omega_c$ как на диаметрах соответственно.
            Каждая из этих окружностей содержит в себе $H$. А также $H_a \in \omega_a, H_b \in \omega_b, H_c \in \omega_c$.

            По \cref{lem:diff Shtainer} $L_aH_a \cap L_bH_b \cap L_cH_c \cap (ABC) = N \neq \varnothing$.

            По \cref{th:miquel's theorem} для треугольника $L_aL_bN$ и точек $H, H_a, H_b$ 

            \begin{equation}
                \underbrace{(HL_aH_a)}_{\omega_a} \cap \underbrace{(HL_bH_b)}_{\omega_b} \cap \underbrace{(H_bH_aN)}_{(ABC)} = M \neq \varnothing.
            \end{equation}

            Аналогично для треугольников $L_aL_cN$ и $L_bL_cN$. Получаем, что 

            \begin{equation}\label{eq:54.2}
                \omega_a \cap \omega_b \cap \omega_b = \{H, M\}
            .\end{equation} 

            По \cref{eq:54.2,def:radaxis} окружности $\omega_a, \omega_b, \omega_c$ имеют общую радикальную ось $MH$. А их центры, точки $M_a, M_b, M_c$, лежат на одной прямой.
        }
    
\end{enumerate}
