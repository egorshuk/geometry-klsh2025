\documentclass[twoside]{article}
\newcommand{\magic}{}
\usepackage{be-my-concrete, be-my-geometry}
\usepackage{extra-geometry}
\usepackage{tasks}

\begin{document}
\pagestyle{empty}

% \input{misc/titlepage.tex}
% \newpage

\begin{abstract}
    Данный курс посвящён решению задач школьной планиметрии. Он будет охватывать
    такие темы как: счёт углов, ортоцентр треугольника, степень точки, движения 
    плоскости, гомотетия и другие. 
    Темы выходят за рамки школьного курса геометрии, поэтому этот курс поможет
    по-новому взглянуть на знакомые темы и задачи, а при решении новых, покажет
    ``незнакомые'' пути решения.

    Курс подойдет для школьников, которые уже знакомы с понятие ``вписанные углы''
    или ``вписанные четырехугольники''.

\end{abstract}

\newpage

\tableofcontents
\newpage

\setcounter{page}{1}
\pagestyle{fancy}

% TODO: не забыть вставить задачки из симплов.

\section{Счет углов}
\input{sections/counting_angles.tex}

% \subsection{Касательные к окружности}
% Задачи на касающиеся объекты периодически попадются на олимпиадах. Могут касаться две окружности, а также прямая и окружность. Чтобы было удобнее обсуждать эту тему изучим еще одну новую технику -- счет в дугах.

\subsubsection{Счёт в дугах}

\begin{lemma}\label{lem:angle-between-chords}
    Две прямые $AB$ и $CD$ пересекаются внутри окружности в точке $P$, а точки $A$, $B$, $C$ и $D$ лежат на окружности. Тогда \[\angle APB = \frac{\overset{\frown}{AC} + \overset{\frown}{BD}}{2}.\]
    Здесь $AB$ и $CD$ -- это меньшие дуги окружности.
\end{lemma}

\begin{figure}[h]
    \centering
    \begin{asy}
        size(5cm);
        point O = (0, 0);
        circle omega = circle(O, 1); draw(omega);

        point A = angpoint(omega, 210);
        point C = angpoint(omega, 100);
        point B = angpoint(omega, 15);
        point D = angpoint(omega, 320);

        point P = intersectionpoint(line(A, B), line(C, D));


        draw(A--B, blue); draw(C--D, blue);
        draw(A--D, red+dashed);
        markangle(n = 3, C, P, A, radius=14, red);
        dot(P);
        markangle(n = 1, C, D, A, radius=14, red);
        markangle(n = 2, D, A, B, radius=14, red);
        dot("$A$", A, dir(210));
        dot("$B$", B, dir(15));
        dot("$C$", C, dir(100));
        dot("$D$", D, dir(320));
        dot("$P$", P, dir(50));
    \end{asy}
    \caption{Угол между хордами.}
    \label{fig:angle-between-chords}
\end{figure}

\begin{lemma}\label{lem:angle-between-secant}
    Две прямые $AB$ и $CD$ пересекаются вне окружности в точке $P$, а точки $A$, $B$, $C$ и $D$ лежат на окружности. Тогда \[\angle APB = \frac{\overset{\frown}{AC} - \overset{\frown}{BD}}{2}.\] Здесь $AB$ и $CD$ -- это меньшие дуги окружности. 
\end{lemma}
\begin{figure}[h]
    \centering
    \begin{asy}
        size(6cm, 5cm);
        point O = (0, 0);
        circle omega = circle(O, 1); draw(omega);

        point A = angpoint(omega, 210);
        point C = angpoint(omega, 90);
        point D = angpoint(omega, 5);
        point B = angpoint(omega, 320);

        point P = intersectionpoint(line(A, B), line(C, D));
        draw(A--P, blue); draw(C--P, blue);
        draw(A--D, red+dashed);
        markangle(n = 3, C, P, A, radius=14, red);
        dot(P);
        markangle(n = 1, C, D, A, radius=14, red);
        markangle(n = 2, B, A, D, radius=14, red);
        dot("$A$", A, dir(210));
        dot("$B$", B, dir(320));
        dot("$C$", C, dir(90));
        dot("$D$", D, dir(5));
        dot("$P$", P);
    \end{asy}
    \caption{Угол между секущими к окружности.}
    \label{fig:angle-between-secant}
\end{figure}

Эта техника и сама бывает стоящей полезной в несложных задачах. Сейчас мы с помощью нее и вычислим угол между касательной и хордой.

\begin{proposition}\label{prop:angle-between-tangent-and-chord}
    Пусть $AB$ -- хорда окружности, а $C$ -- точка касания касательной к окружности. Тогда угол между касательной и хордой равен вписанному углу, операющему на ту же дугу, что и хорда. То есть \[\angle ACB = \frac{\overset{\frown}{AB}}{2}.\]
\end{proposition}
\begin{proof}
    Пусть $P$ -- точка пересечения касательной с прямой $AB$. Тогда можем посчитать угол $\angle APC$ по \cref{lem:angle-between-secant}:
    \[
        \angle APC = \frac{\overset{\frown}{AC} - \overset{\frown}{BC}}{2}.
    \]

\end{proof}

\begin{figure}
    \centering
    \begin{asy}
        size(6cm, 5cm);
        point O = (0, 0);
        circle omega = circle(O, 1); draw(omega);

        point A = angpoint(omega, 210);
        point B = angpoint(omega, 70);
        point C = angpoint(omega, 5);

        line t = tangent(omega, C);
        point P = intersectionpoint(line(A, B), t);

        draw(t, red);
        draw(A--B);
        draw(C--A);
        draw(C--B, blue);
        draw(B--P, dashed);

        markangle(n = 1, t, line(B, C), radius=20, red);
        markangle(n = 2, C, A, B, radius=20, red);
        markangle(n = 3, A, P, C, radius=20, blue);

        dot("$A$", A, dir(210));
        dot("$B$", B, dir(90));
        dot("$C$", C, dir(5));
        dot("$P$", P, dir(40));
    \end{asy}
    \caption{Угол между касательной и хордой.}
    \label{fig:angle-between-tangent-and-chord}
\end{figure}


\section{Площади}
\subsection{Площади простых фигур}

\section{Ортоцентр треугольника}
\input{sections/symmetry_orthocenter.tex}
\input{sections/other_orthocenter.tex}
\section{Популярные уголки}


\section{Подобие треугольников}

% придумать много простых

\section{Симметрии}

\section{Степень точки}
\input{sections/pow.tex}

\subsection{Радикальная ось}
\input{sections/radical.tex}

\newpage
\renewcommand{\thesubsection}{\roman{subsection}}
\setcounter{subsection}{0}

\section*{Задачи}
\addcontentsline{toc}{section}{Задачи}
\subsection{Счёт углов-I}
\input{tasks/counting_angles.tex}
\subsection{Площади}
% % TODO: написать решения

\begin{tasks}
    \item\label{task:easy-varinion} Площадь прямоугольника равна 24. Найдите площадь четырехугольника с вершинами в серединах сторон прямоугольника.
    \item Средняя линия треугольника разбивает его на треугольник и четырехугольник. Какую часть составляет площадь полученного треугольника от площади исходного?
    \item Точка $M$ расположена на стороне $BC$ параллелограмма $ABCD$. Докажите, что площадь треугольника $AMD$ равна половине площади параллелограмма.
    \item Пусть $M$ -- точка на стороне $AB$ треугольника $ABC$, причем $AM : MB = m : n$. Докажите, что площадь треугольника $CAM$ относится к площади треугольника $CBM$ как $m : n$.
    \item Точки $M$ и $N$ -- соотвественно середины противоположных сторон $AB$ и $CD$ параллелограмма $ABCD$, площадь которого равна 1. Найдите площадь четырехугольника, образованного пересечениями прымях $AN$, $BN$, $CM$, $DM$.
    \item На сторонах $AB$ и $AC$ треугольника $ABC$, площадь которого равна 50, взяты соответственно точки $M$ и $K$ так, что $AM : MB = 1 : 5$, а $AK : KC = 3 : 2$. Найдите площадь треугольника $AMK$.
    \item Прямая, проведенная через вершину $C$ трапеции $ABCD$ параллельно диагонали $BD$, пересекает продолжение основания $AD$ в точке $M$. Докажите, что треугольник $ACM$ равновелик трапеции $ABCD$.
    \item Докажите, что медианы треугольника делят его на шесть равновеликих частей.
    \item Медианы $BM$ и $CN$ треугольника $ABC$ пересекаются в точке $K$. Докажите, что четырехугольник $AMKN$ равновелик треугольнику $BKC$.
    \item Точка внутри параллелограмма соединена со всеми его вершинами. Докажите, что суммы площадей треугольников, прилежащих к противоположным сторонам параллелограмма, равны между собой.
    \item Середины сторон выпуклого четырехугольника последовательно соединены отрезками. Докажите, что площадь полученного четырехугольника вдвое меньше площади исходного.\footnote{Привет \cref{task:easy-varinion}!}
    \item Отрезки, соединяющие середины противоположных сторон выпуклого четырехугольника, взаимно перпендикулярны и равны 2 и 7. Найдите площадь четырехугольника.
        \moditem{*} Докажите, что сумма расстояний от произвольной точки внутри равностороннего треугольника до его сторон всегда одна и та же.
    \item Докажите, что площадь треугольника равна произведению полупериметра треугольника и радиуса вписанной окружности.
    \item Дан треугольник $ABC$. Найдите геометрическое место таких точек $M$, для которых:
        \begin{tasks}
        \item треугольники $AMB$ и $ABC$ равновелики;
        \item треугольники $AMB$ и $AMC$ равновелики;
        \item треугольники $AMB$, $AMC$ и $BMC$ равновелики.
        \end{tasks}
    \item Боковая сторона $AB$ и основание $BC$ трапеции $ABCD$ вдвое меньше ее основания $AD$. Найдите площадь трапеции, если $AC = a$, $CD = b$.

        % тут если будет слишко легко -- дать просто задачи 3 уровня из гордина. 
\end{tasks}

\subsection{Счёт углов-II}
\input{tasks/orthocenter.tex}
\subsection{Подобие}
% \input{tasks/simirality.tex}
\subsection{Симметрия}
% \begin{tasks}
    \item Пусть $M$ и $N$  --- середины оснований трапеции. Докажите, что если прямая $MN$ перпендикулярна основаниям, то трапеция равнобедренная.
    \item Пусть $M$ --- середина отрезка $AB$. Точки $A'$, $B'$ и $M'$ --- образы точек соответственно $A$, $B$ и $M$ при симметрии относительно некоторой точки $O$. Докажите, что $M'$ --- середина $A'B'$.
    \item На противоположных сторонах параллелограмма как на сторонах построены вне параллелограмма два квадрата. Докажите, что прямая, соединяющая их центры, проходит через центр параллелограмма.
    \item Докажите, что точки, симметричные произвольной точке относительно середин сторон квадрата, являются вершинами некоторого квадрата.
    \item Даны две концентрические окружности $S_1$ и $S_2$. Постройте прямую, на которой эти окружности высекают три равных отрезка.
    \item Противоположные стороны выпуклого шестиугольника попарно равны и параллельны. Докажите, что он имеет центр симметрии.
    \item Диагонали $AC$ и $BD$ параллелограмма $ABCD$ пересекаются в точке $O$. Докажите, что окружности, описанные около треугольников $AOB$ и $COD$, касаются
    \item Фигура имеет две перпендикулярные оси симметрии. Докажите, что она имеет центр симметрии.
    \item Точки $A$ и $B$ лежат по разные стороны от прямой $\ell$. Постройте на этой прямой точку $M$ так, чтобы прямая $\ell$ делила угол $AMB$ пополам.
    \item Внутри острого угла даны точки $M$ и $N$. Как из точки $M$ направить луч света, чтобы он, отразившись последовательно от сторон угла, попал в точку $N$?

\end{tasks}

\subsection{Степень точки и радикальная ось}
\input{tasks/pow_radical.tex}

\newpage
\addcontentsline{toc}{section}{Контрольная работа}
\section*{Контрольная работа}
% \input{tasks/exam.tex}

\end{document}
