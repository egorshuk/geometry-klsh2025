\documentclass[twoside]{article}
\newcommand{\magic}{S}
\usepackage{be-my-concrete, be-my-geometry}
\usepackage{extra-geometry}
\input{./tucyradd.def}

\begin{document}
\pagestyle{empty}

% \input{misc/titlepage.tex}
% \newpage

\begin{abstract}
    Данный курс посвящён решению задач школьной планиметрии. Он будет охватывать
    такие темы как: счёт углов, ортоцентр треугольника, степень точки, движения 
    плоскости, гомотетия и другие. 
    Темы выходят за рамки школьного курса геометрии, поэтому этот курс поможет
    по-новому взглянуть на знакомые темы и задачи, а при решении новых, покажет
    ``незнакомые'' пути решения.

    Курс подойдет для школьников, которые уже знакомы с понятие ``вписанные углы''
    или ``вписанные четырехугольники''.

\end{abstract}

\newpage

\tableofcontents
\newpage

\setcounter{page}{1}
\pagestyle{fancy}


\section{Счет углов}
\input{sections/counting_angles.tex}

% \subsection{Касательные к окружности}
% Задачи на касающиеся объекты периодически попадются на олимпиадах. Могут касаться две окружности, а также прямая и окружность. Чтобы было удобнее обсуждать эту тему изучим еще одну новую технику -- счет в дугах.

\subsubsection{Счёт в дугах}

\begin{lemma}\label{lem:angle-between-chords}
    Две прямые $AB$ и $CD$ пересекаются внутри окружности в точке $P$, а точки $A$, $B$, $C$ и $D$ лежат на окружности. Тогда \[\angle APB = \frac{\overset{\frown}{AC} + \overset{\frown}{BD}}{2}.\]
    Здесь $AB$ и $CD$ -- это меньшие дуги окружности.
\end{lemma}

\begin{figure}[h]
    \centering
    \begin{asy}
        size(5cm);
        point O = (0, 0);
        circle omega = circle(O, 1); draw(omega);

        point A = angpoint(omega, 210);
        point C = angpoint(omega, 100);
        point B = angpoint(omega, 15);
        point D = angpoint(omega, 320);

        point P = intersectionpoint(line(A, B), line(C, D));


        draw(A--B, blue); draw(C--D, blue);
        draw(A--D, red+dashed);
        markangle(n = 3, C, P, A, radius=14, red);
        dot(P);
        markangle(n = 1, C, D, A, radius=14, red);
        markangle(n = 2, D, A, B, radius=14, red);
        dot("$A$", A, dir(210));
        dot("$B$", B, dir(15));
        dot("$C$", C, dir(100));
        dot("$D$", D, dir(320));
        dot("$P$", P, dir(50));
    \end{asy}
    \caption{Угол между хордами.}
    \label{fig:angle-between-chords}
\end{figure}

\begin{lemma}\label{lem:angle-between-secant}
    Две прямые $AB$ и $CD$ пересекаются вне окружности в точке $P$, а точки $A$, $B$, $C$ и $D$ лежат на окружности. Тогда \[\angle APB = \frac{\overset{\frown}{AC} - \overset{\frown}{BD}}{2}.\] Здесь $AB$ и $CD$ -- это меньшие дуги окружности. 
\end{lemma}
\begin{figure}[h]
    \centering
    \begin{asy}
        size(6cm, 5cm);
        point O = (0, 0);
        circle omega = circle(O, 1); draw(omega);

        point A = angpoint(omega, 210);
        point C = angpoint(omega, 90);
        point D = angpoint(omega, 5);
        point B = angpoint(omega, 320);

        point P = intersectionpoint(line(A, B), line(C, D));
        draw(A--P, blue); draw(C--P, blue);
        draw(A--D, red+dashed);
        markangle(n = 3, C, P, A, radius=14, red);
        dot(P);
        markangle(n = 1, C, D, A, radius=14, red);
        markangle(n = 2, B, A, D, radius=14, red);
        dot("$A$", A, dir(210));
        dot("$B$", B, dir(320));
        dot("$C$", C, dir(90));
        dot("$D$", D, dir(5));
        dot("$P$", P);
    \end{asy}
    \caption{Угол между секущими к окружности.}
    \label{fig:angle-between-secant}
\end{figure}

Эта техника и сама бывает стоящей полезной в несложных задачах. Сейчас мы с помощью нее и вычислим угол между касательной и хордой.

\begin{proposition}\label{prop:angle-between-tangent-and-chord}
    Пусть $AB$ -- хорда окружности, а $C$ -- точка касания касательной к окружности. Тогда угол между касательной и хордой равен вписанному углу, операющему на ту же дугу, что и хорда. То есть \[\angle ACB = \frac{\overset{\frown}{AB}}{2}.\]
\end{proposition}
\begin{proof}
    Пусть $P$ -- точка пересечения касательной с прямой $AB$. Тогда можем посчитать угол $\angle APC$ по \cref{lem:angle-between-secant}:
    \[
        \angle APC = \frac{\overset{\frown}{AC} - \overset{\frown}{BC}}{2}.
    \]

\end{proof}

\begin{figure}
    \centering
    \begin{asy}
        size(6cm, 5cm);
        point O = (0, 0);
        circle omega = circle(O, 1); draw(omega);

        point A = angpoint(omega, 210);
        point B = angpoint(omega, 70);
        point C = angpoint(omega, 5);

        line t = tangent(omega, C);
        point P = intersectionpoint(line(A, B), t);

        draw(t, red);
        draw(A--B);
        draw(C--A);
        draw(C--B, blue);
        draw(B--P, dashed);

        markangle(n = 1, t, line(B, C), radius=20, red);
        markangle(n = 2, C, A, B, radius=20, red);
        markangle(n = 3, A, P, C, radius=20, blue);

        dot("$A$", A, dir(210));
        dot("$B$", B, dir(90));
        dot("$C$", C, dir(5));
        dot("$P$", P, dir(40));
    \end{asy}
    \caption{Угол между касательной и хордой.}
    \label{fig:angle-between-tangent-and-chord}
\end{figure}


\section{Симметрии}

\section{Площади}
Давайте сейчас повторим то, чего мы уже знаем про площади различных многоугольников.

\begin{itemize}
    \item Равные многоулольники имеют равную площадь.
    \item Если многоугольник составлен из нескольких многоугольников, то его площадь равняется сумме площадей этих многоугольников.
    \item Площадь прямогоульника равняется прозведению двух его сторон.
    \item Площадь треугольника равняется половине произведения его основания на высоту.
    \item Площадь параллелограмма равна произведению его основания на высоту.
    \item Площадь ромба равна половине произведения его диагоналей
    \item Площадь трапеции равна произведению ее средней линии на высоту.
    \item Фигуры, имеющие равные площади, называются равновеликими. 
\end{itemize}

% TODO: нужно сделать картинки для всего-всего тут.

\begin{example}
    Докажите, что в треугольнике высоты обратно пропорциональна сторонам, к которым они проведены.
\end{example}

\begin{example}
    Медины $AA_1$ и $BB_1$ треугольника $ABC$ пересекаются в точке $M$. Найдите площадь треугольника $A_1MB_1$, если площадь треугольника $ABC$ равняется 1.
\end{example}

\begin{example}
    Докажите, что биссектриса треугольника делит его сторону на отрезки, пропорциональные двум другим сторонам.
\end{example}


\section{Ортоцентр треугольника}
\input{sections/symmetry_orthocenter.tex}
\input{sections/other_orthocenter.tex}
\subsection{Другие популярные уголки}
\begin{lemma}\label{lem:con-or-kite}
    Если в четырехугольнике $ABCD$, $AC$ -- биссектриса угла $A$ и $BC = CD$, то этот четырехугольник является либо вписанным, либо дельтойдом\footnote{\emph{Дельтойдом (кайтом)} называется четырехугольик, у которого есть две пары равных смежных сторон}.
\end{lemma}

\begin{figure}[h]
    \centering
    \begin{asy}
        size(6cm, 5cm);
        triangle t = triangleabc(13, 9, 10);

        circle omega = circle(t); 
        point M = intersectionpoints(omega, bisector(t.VC))[0];

        draw(point(t.VA)--point(t.VC)--point(t.VB)--M--cycle, linewidth(bp));

        draw(point(t.VC)--M, blue);
        markangle(n = 1, point(t.VA), point(t.VC), M, radius=16, red);
        markangle(n = 1, M, point(t.VC), point(t.VB), radius=18, red);

        draw(point(t.VA)--M, StickIntervalMarker(1, 1, size=7));
        draw(point(t.VB)--M, StickIntervalMarker(1, 1, size=7));

        point H = intersectionpoint(t.BC, perpendicular(t.VA, line(t.VC, M)));
        draw(M--H, dashed+red, StickIntervalMarker(1, 1, size=7));
        dot(H);

        draw(omega, red+dashed);

        label("$A$", point(t.VC), N);
        label("$B$", point(t.VB), E);
        label("$C$", M, S);
        label("$D$", point(t.VA), W);
    \end{asy}
    \label{fig:con-or-kite}
    \caption{Случай вписанности и дельтойда.}
\end{figure}


\section{Подобие треугольников} 

% придумать много простых


\section{Степень точки}
\input{sections/pow.tex}

\subsection{Радикальная ось}
\input{sections/radical.tex}

\newpage
\renewcommand{\thesubsection}{\roman{subsection}}
\setcounter{subsection}{0}

\section*{Задачи}
\markboth{Задачи}{}
\addcontentsline{toc}{section}{Задачи}
% TODO: не забыть вставить задачки из симплов.
% TODO: вставить и посмотреть задачи из 2.10 Гордина.
% TODO: можно брать задачи из разминок всяких каналов!
\subsection{Счёт углов-I}
\input{tasks/counting_angles.tex}
\subsection{Симметрия}  % TODO: взять из параграфа 2.6 Гордина
% \begin{tasks}
    \item Пусть $M$ и $N$  --- середины оснований трапеции. Докажите, что если прямая $MN$ перпендикулярна основаниям, то трапеция равнобедренная.
    \item Пусть $M$ --- середина отрезка $AB$. Точки $A'$, $B'$ и $M'$ --- образы точек соответственно $A$, $B$ и $M$ при симметрии относительно некоторой точки $O$. Докажите, что $M'$ --- середина $A'B'$.
    \item На противоположных сторонах параллелограмма как на сторонах построены вне параллелограмма два квадрата. Докажите, что прямая, соединяющая их центры, проходит через центр параллелограмма.
    \item Докажите, что точки, симметричные произвольной точке относительно середин сторон квадрата, являются вершинами некоторого квадрата.
    \item Даны две концентрические окружности $S_1$ и $S_2$. Постройте прямую, на которой эти окружности высекают три равных отрезка.
    \item Противоположные стороны выпуклого шестиугольника попарно равны и параллельны. Докажите, что он имеет центр симметрии.
    \item Диагонали $AC$ и $BD$ параллелограмма $ABCD$ пересекаются в точке $O$. Докажите, что окружности, описанные около треугольников $AOB$ и $COD$, касаются
    \item Фигура имеет две перпендикулярные оси симметрии. Докажите, что она имеет центр симметрии.
    \item Точки $A$ и $B$ лежат по разные стороны от прямой $\ell$. Постройте на этой прямой точку $M$ так, чтобы прямая $\ell$ делила угол $AMB$ пополам.
    \item Внутри острого угла даны точки $M$ и $N$. Как из точки $M$ направить луч света, чтобы он, отразившись последовательно от сторон угла, попал в точку $N$?

\end{tasks}

\subsection{Площади}
% TODO: написать решения

\begin{tasks}
    \item\label{task:easy-varinion} Площадь прямоугольника равна 24. Найдите площадь четырехугольника с вершинами в серединах сторон прямоугольника.
    \item Средняя линия треугольника разбивает его на треугольник и четырехугольник. Какую часть составляет площадь полученного треугольника от площади исходного?
    \item Точка $M$ расположена на стороне $BC$ параллелограмма $ABCD$. Докажите, что площадь треугольника $AMD$ равна половине площади параллелограмма.
    \item Пусть $M$ -- точка на стороне $AB$ треугольника $ABC$, причем $AM : MB = m : n$. Докажите, что площадь треугольника $CAM$ относится к площади треугольника $CBM$ как $m : n$.
    \item Точки $M$ и $N$ -- соотвественно середины противоположных сторон $AB$ и $CD$ параллелограмма $ABCD$, площадь которого равна 1. Найдите площадь четырехугольника, образованного пересечениями прымях $AN$, $BN$, $CM$, $DM$.
    \item На сторонах $AB$ и $AC$ треугольника $ABC$, площадь которого равна 50, взяты соответственно точки $M$ и $K$ так, что $AM : MB = 1 : 5$, а $AK : KC = 3 : 2$. Найдите площадь треугольника $AMK$.
    \item Прямая, проведенная через вершину $C$ трапеции $ABCD$ параллельно диагонали $BD$, пересекает продолжение основания $AD$ в точке $M$. Докажите, что треугольник $ACM$ равновелик трапеции $ABCD$.
    \item Докажите, что медианы треугольника делят его на шесть равновеликих частей.
    \item Медианы $BM$ и $CN$ треугольника $ABC$ пересекаются в точке $K$. Докажите, что четырехугольник $AMKN$ равновелик треугольнику $BKC$.
    \item Точка внутри параллелограмма соединена со всеми его вершинами. Докажите, что суммы площадей треугольников, прилежащих к противоположным сторонам параллелограмма, равны между собой.
    \item Середины сторон выпуклого четырехугольника последовательно соединены отрезками. Докажите, что площадь полученного четырехугольника вдвое меньше площади исходного.\footnote{Привет \cref{task:easy-varinion}!}
    \item Отрезки, соединяющие середины противоположных сторон выпуклого четырехугольника, взаимно перпендикулярны и равны 2 и 7. Найдите площадь четырехугольника.
        \moditem{*} Докажите, что сумма расстояний от произвольной точки внутри равностороннего треугольника до его сторон всегда одна и та же.
    \item Докажите, что площадь треугольника равна произведению полупериметра треугольника и радиуса вписанной окружности.
    \item Дан треугольник $ABC$. Найдите геометрическое место таких точек $M$, для которых:
        \begin{tasks}
        \item треугольники $AMB$ и $ABC$ равновелики;
        \item треугольники $AMB$ и $AMC$ равновелики;
        \item треугольники $AMB$, $AMC$ и $BMC$ равновелики.
        \end{tasks}
    \item Боковая сторона $AB$ и основание $BC$ трапеции $ABCD$ вдвое меньше ее основания $AD$. Найдите площадь трапеции, если $AC = a$, $CD = b$.

        % тут если будет слишко легко -- дать просто задачи 3 уровня из гордина. 
\end{tasks}

\subsection{Счёт углов-II}
\input{tasks/orthocenter.tex}
\begin{tasks}
    \item Докажите, что точка, симметричная точке пересечения высот (ортоцентру) треугольника относительно стороны, лежит на описанной окружности этого треугольника.

    \item Пусть точка $O$ --- центр описанной окружности треугольника $ABC$, $AH$ --- высота. Докажите, что $\angle BAH = \angle OAC$.
        
    \item Пусть $AA_1$ и $BB_1$ --- высоты остроугольного треугльника $ABC$, а точка $O$ --- центр его описанной окружности. Докажите, что $CO \perp A_1B_1$.

    \item В треугольнике $ABC$ проведены высоты $BB_1$ и $CC_1$, а также отмечена точка $M$ --- середина стороны $BC$. Точка $H$ --- его ортоцентр, а точка $P$ --- пересечения луча \texttt{(!)} $MH$ с окружностью $(ABC)$. Докажите, что точки $P, A, B_1, C_1$  концикличны. 

    \item Во вписанном четырехугольнике $ABCD$ точка $P$ --- точка пересечения диагоналей $AC$ и $BD$. Точка $O$ --- центр окружности $(ABP)$. Докажите, что $OP \perp CD$. 

    \item \named{Муниципальный этап ВСОШ (Москва), 2020, 9.4}{Пусть точки $B$ и $C$ лежат на по\-лу\-окруж\-но\-сти с диаметром $AD$. Точка $M$ --- середина отрезка $BC$. Точка $N$ такова, что точка $M$ --- середина отрезка $AN$, докажите что $BC \perp ND$}. 

    \item В треугольнике $ABC$ проведена высота $AD$ и отмечен центр описанной окружности --- $O$. Пусть точки $E$ и $F$ --- проекции точек $B$ и $C$ на прямую $AO$. $N$ --- точка пересечения прямых $AC$ и $DE$, а $M$ --- точка пересечения прямых $AB$ и $DF$. Докажите, что точки $A, D, N, M$ концикличны.

    \item Окружность $S_2$ проходит через центр $O$ окружности $S_1$ и пересекает ее в точках $A$ и $B$. Через точку A проведена касательная к окружности $S_2$; $D$ --- вторая точка пересечения этой касательной с окружностью $S_1$. Докажите, что $AD = AB$.

    \item \named{Baltic Way, 2019, problem 12}{Let $ABC$ be a triangle and $H$ its orthocenter. Let $D$ be a point lying on the segment $AC$ and let $E$ be the point on the line $BC$ such that $BC \perp DE$. Prove that $EH \perp BD$ if and only if $BD$ bisects $AE$}. 

    \item \named{Лемма Архимеда}{Две окружности касаются внутренним образом в точке $M$. Пусть $AB$ --- хорда большей окружности, касающаяся меньшей окружности в точке $T$. Докажите, что $MT$ ---  биссектриса угла $AMB$.}

    \item В трапеции $ABCD$ с основаниями $AB$ $CD$ выполнено равенство  $AB = BD+CD$. Пусть $𝐸$ --- середина $𝐴𝐶$. Докажите, что $\angle BED = 90^\circ$.

    \item В параллелограмме $ABCD$ диагональ $AC$ больше диагонали $BD$. Точка $M$ на диагонали $AC$ такова, что около четырехугольника $BCDM$ можно описать окружность. Докажите, что $BD$ --- общая касательная окружностей, описанных около треугольников $ABM$ и $ADM$.

    \item \named{Прямая Симсона}{Докажите, что основания перпендикуляров, опущенных из произвольной точки описанной окружности на стороны треугольника (или их продолжения), лежат на одной прямой.}

    \item Пусть $H$ --- ортоцентр остроугольного треугольника $ABC$ Серединный перпендикуляр $\ell$ к стороне $AC$ пересекает прямые $AH$, $CH$ в точках $K$ и $L$ соответственно. Докажите, что ортоцентр треугольника  лежит на прямой, содержащей одну из медиан треугольника $ABC$.

    \end{tasks}

\subsection{Подобие}    % TODO: взять из параграфа 2.9 Гордина
% \input{tasks/simirality.tex}
\subsection{Степень точки и радикальная ось}    %TODO: взять задач из 3.1 Гордина
\input{tasks/pow_radical.tex}

\newpage
\markboth{Контрольная работа}{}
\addcontentsline{toc}{section}{Контрольная работа}
\section*{Контрольная работа}
% \input{tasks/exam.tex}

\end{document}
