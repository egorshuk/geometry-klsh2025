\documentclass[twoside]{article}
\newcommand{\magic}{S}
\usepackage{be-my-concrete, be-my-geometry}
\usepackage{extra-geometry}
% taken from xecyr.sty
%%% Cyrillic letter TeX definitions for XeTeX (and LuaTeX)
% This definition set is complete for all Slavic Cyrillic 
% languages (Russian, Ukrainian, Belorussian,
% Rusyn, Serbian, Macedonian and Bulgarian alphabets).
%
% It has not been confirmed as complete, but might be, for those
% non-Slavic Cyrillic languages:
% Ossetian, Khalka, Buryat, Kalmyk, Kyrgyz, Tatar, Uzbek
% Azerbaijani, Kazakh, Abkhaz, Chukchi.

%%% Part I
%%% Cyrillic letters defined as a continuous stretch 
%%% in Unicode and most common 8-bit encodings which define them:

\DeclareTextSymbol{\CYRA}\UnicodeEncodingName{"0410}        % А
\DeclareTextSymbol{\cyra}\UnicodeEncodingName{"0430}        % а
\DeclareTextSymbol{\CYRB}\UnicodeEncodingName{"0411}        % Б
\DeclareTextSymbol{\cyrb}\UnicodeEncodingName{"0431}        % б
\DeclareTextSymbol{\CYRV}\UnicodeEncodingName{"0412}        % В 
\DeclareTextSymbol{\cyrv}\UnicodeEncodingName{"0432}        % в
\DeclareTextSymbol{\CYRG}\UnicodeEncodingName{"0413}        % Г
\DeclareTextSymbol{\cyrg}\UnicodeEncodingName{"0433}        % г
\DeclareTextSymbol{\CYRD}\UnicodeEncodingName{"0414}        % Д
\DeclareTextSymbol{\cyrd}\UnicodeEncodingName{"0434}        % д
\DeclareTextSymbol{\CYRE}\UnicodeEncodingName{"0415}        % Е 
\DeclareTextSymbol{\cyre}\UnicodeEncodingName{"0435}        % е
\DeclareTextSymbol{\CYRZH}\UnicodeEncodingName{"0416}       % Ж 
\DeclareTextSymbol{\cyrzh}\UnicodeEncodingName{"0436}       % ж
\DeclareTextSymbol{\CYRZ}\UnicodeEncodingName{"0417}        % З
\DeclareTextSymbol{\cyrz}\UnicodeEncodingName{"0437}        % з
\DeclareTextSymbol{\CYRI}\UnicodeEncodingName{"0418}        % И
\DeclareTextSymbol{\cyri}\UnicodeEncodingName{"0438}        % и
\DeclareTextSymbol{\CYRISHRT}\UnicodeEncodingName{"0419}    % Й
\DeclareTextSymbol{\cyrishrt}\UnicodeEncodingName{"0439}    % й
\DeclareTextSymbol{\CYRK}\UnicodeEncodingName{"041A}        % К
\DeclareTextSymbol{\cyrk}\UnicodeEncodingName{"043A}        % к
\DeclareTextSymbol{\CYRL}\UnicodeEncodingName{"041B}        % Л
\DeclareTextSymbol{\cyrl}\UnicodeEncodingName{"043B}        % л 
\DeclareTextSymbol{\CYRM}\UnicodeEncodingName{"041C}        % М
\DeclareTextSymbol{\cyrm}\UnicodeEncodingName{"043C}        % м
\DeclareTextSymbol{\CYRN}\UnicodeEncodingName{"041D}        % Н
\DeclareTextSymbol{\cyrn}\UnicodeEncodingName{"043D}        % н
\DeclareTextSymbol{\CYRO}\UnicodeEncodingName{"041E}        % О
\DeclareTextSymbol{\cyro}\UnicodeEncodingName{"043E}        % о
\DeclareTextSymbol{\CYRP}\UnicodeEncodingName{"041F}        % П
\DeclareTextSymbol{\cyrp}\UnicodeEncodingName{"043F}        % п
\DeclareTextSymbol{\CYRR}\UnicodeEncodingName{"0420}        % Р
\DeclareTextSymbol{\cyrr}\UnicodeEncodingName{"0440}        % р
\DeclareTextSymbol{\CYRS}\UnicodeEncodingName{"0421}        % С
\DeclareTextSymbol{\cyrs}\UnicodeEncodingName{"0441}        % с
\DeclareTextSymbol{\CYRT}\UnicodeEncodingName{"0422}        % Т
\DeclareTextSymbol{\cyrt}\UnicodeEncodingName{"0442}        % т
\DeclareTextSymbol{\CYRU}\UnicodeEncodingName{"0423}        % У
\DeclareTextSymbol{\cyru}\UnicodeEncodingName{"0443}        % у
\DeclareTextSymbol{\CYRF}\UnicodeEncodingName{"0424}        % Ф
\DeclareTextSymbol{\cyrf}\UnicodeEncodingName{"0444}        % ф
\DeclareTextSymbol{\CYRH}\UnicodeEncodingName{"0425}        % Х
\DeclareTextSymbol{\cyrh}\UnicodeEncodingName{"0445}        % х
\DeclareTextSymbol{\CYRC}\UnicodeEncodingName{"0426}        % Ц
\DeclareTextSymbol{\cyrc}\UnicodeEncodingName{"0446}        % ц
\DeclareTextSymbol{\CYRCH}\UnicodeEncodingName{"0427}       % Ч
\DeclareTextSymbol{\cyrch}\UnicodeEncodingName{"0447}       % ч
\DeclareTextSymbol{\CYRSH}\UnicodeEncodingName{"0428}       % Ш
\DeclareTextSymbol{\cyrsh}\UnicodeEncodingName{"0448}       % ш
\DeclareTextSymbol{\CYRSHCH}\UnicodeEncodingName{"0429}     % Щ
\DeclareTextSymbol{\cyrshch}\UnicodeEncodingName{"0449}     % щ
\DeclareTextSymbol{\CYRHRDSN}\UnicodeEncodingName{"042A}    % Ъ
\DeclareTextSymbol{\cyrhrdsn}\UnicodeEncodingName{"044A}    % ъ
\DeclareTextSymbol{\CYRERY}\UnicodeEncodingName{"042B}      % Ы
\DeclareTextSymbol{\cyrery}\UnicodeEncodingName{"044B}      % ы
\DeclareTextSymbol{\CYRSFTSN}\UnicodeEncodingName{"042C}    % Ь
\DeclareTextSymbol{\cyrsftsn}\UnicodeEncodingName{"044C}    % ь
\DeclareTextSymbol{\CYREREV}\UnicodeEncodingName{"042D}     % Э
\DeclareTextSymbol{\cyrerev}\UnicodeEncodingName{"044D}     % э
\DeclareTextSymbol{\CYRYU}\UnicodeEncodingName{"042E}       % Ю
\DeclareTextSymbol{\cyryu}\UnicodeEncodingName{"044E}       % ю
\DeclareTextSymbol{\CYRYA}\UnicodeEncodingName{"042F}       % Я
\DeclareTextSymbol{\cyrya}\UnicodeEncodingName{"044F}       % я

%%% Part II
%%% Cyrillic letters not defined as continuous stretches,
%%% but available in common 8-bit cyrillic encodings,
%%% like cp1251 and iso-8859-5, sorted by languages 
%%% they're used in for ease of checking:

% Common to many languages:
\DeclareTextSymbol{\CYRYO}\UnicodeEncodingName{"0401}       % Ё
\DeclareTextSymbol{\cyryo}\UnicodeEncodingName{"0451}       % ё

\endinput


\begin{document}
\pagestyle{empty}

% \input{misc/titlepage.tex}
% \newpage

\begin{abstract}
    Данный курс посвящён решению задач школьной планиметрии. Он будет охватывать
    такие темы как: счёт углов, ортоцентр треугольника, степень точки, движения 
    плоскости, гомотетия и другие. 
    Темы выходят за рамки школьного курса геометрии, поэтому этот курс поможет
    по-новому взглянуть на знакомые темы и задачи, а при решении новых, покажет
    ``незнакомые'' пути решения.

    Курс подойдет для школьников, которые уже знакомы с понятие ``вписанные углы''
    или ``вписанные четырехугольники''.

\end{abstract}

\newpage

\tableofcontents
\newpage

\setcounter{page}{1}
\pagestyle{fancy}


\section{Счет углов}
Под этим названием скрывается, не побоюсь этого слова, самый (!) используемый метод в решении задач. В каждой он встречается в том или ином виде. Поэтому, если вы хотите решать задачи, вам нужно его знать. В основном ``считаются'' углы, связанные с окружностями, но бывает и что-то другое.

Для примера, давайте дакажем, что высоты треугольника пересекаются в одной точке. Для этого вспомним <<вписанные углы>>
\begin{theorem}\label{th:otrhocenter}
    Высоты треугольника \emph{конкурентны\footnote{Пересекаются в одной точке.}}.
\end{theorem}
\begin{figure}[ht]
    \centering
    \begin{asy}
        size(9cm);
        triangle t = triangleabc(9, 10, 11); draw(t, linewidth(bp));

        point H_a = foot(t.VA); point H_b = foot(t.VB); point H_c = foot(t.VC);
        point H = orthocentercenter(t);

        draw(segment(t.VA, H_a), grey); draw(segment(t.VB, H_b), grey); draw(segment(t.VC, H_c), grey);

        perpendicularmark(t.AB, line(t.VC, H_c), dashed + blue, size=10);
        perpendicularmark(t.BC, line(t.VA, H_a), blue, size=10, quarter=1);
        perpendicularmark(t.AC, line(t.VB, H_b), blue, size=10, quarter=3);

        draw(circle(t.VC, H_a, H_b));
        draw(H_a--H_b, grey);
        
        markangle(n = 1, H_a, H, point(t.VC), radius=12, red);
        markangle(n = 1, H_a, H_b, point(t.VC), radius=14, red);

        clipdraw(circle(t.VA, H_a, H_b));

        //draw(arc(circle(t.VA, H_a, H_b), -5, 185));

        markangle(n = 1, H_a, point(t.VB), H_c, radius=16, red);

        draw(circle(H, H_a, H_c), dashed+red+bp*0.7);
    \end{asy}
    \caption{Высоты треугольника пересекаются в одной точке.}
    \label{fig:orthocenter}
\end{figure}

\noindent
\begin{minipage}{0.65\linewidth}
    \begin{lemma}\label{lem:concycle}
        Четырехугольник $ABCD$ является вписанным, если $\angle ABC$ равен смежному углу $\angle ADC$.
    \end{lemma}
\end{minipage}
\hspace{0.05\linewidth}
\begin{minipage}{0.3\linewidth}
    \begin{asy}
        size(3.5cm);
        point O = (0, 0);
        circle omega = circle(O, 1); draw(omega);

        point A = angpoint(omega, 210);
        point B = angpoint(omega, 100);
        point C = angpoint(omega, 15);
        point D = angpoint(omega, 320);

        point P = intersectionpoints(circle(B, 2), line(A, B))[0];
        
        draw(P--A--B--C--D--A);
        markangle(n = 1, line(B, A), line(A, D), radius=12, red+dashed);
        markangle(n = 1, B, C, D, radius=14, red);

        markangle(n = 2, D, A, B, radius=14, blue);
    \end{asy}
\end{minipage}

\begin{proposition}\label{prop:angle-between-tangent-and-chord}
    Пусть $AB$ -- хорда окружности, а $C$ -- точка касания касательной к окружности. Тогда угол между касательной и хордой равен вписанному углу, операющему на ту же дугу, что и хорда. То есть \(\angle ACB = \frac{\overset{\frown}{AB}}{2}.\)
\end{proposition}
% \begin{proof}
%     Пусть $O$ -- центр окружности. Тогда отрезки $OB$ и $ОС$ равны как радиусы. При том, угол $\angle BOC = 2\cdot \angle BAC$. Радиус $OC$ перпенидкулярен касательной в точке $C$. Значит угол между касательной и хордой равен:
%     \[
%         90^\circ - \frac{180^\circ - 2\cdot\angle BAC}{2} = \angle BAC.
%     \]
% \end{proof}

\begin{figure}[ht]
    \centering
    \begin{asy}
        size(6cm, 5cm);
        point O = (0, 0);
        circle omega = circle(O, 1); draw(omega);

        point A = angpoint(omega, 210);
        point B = angpoint(omega, 70);
        point C = angpoint(omega, 5);

        line t = tangent(omega, C); draw(t, red);

        draw(A--B);
        draw(C--A);
        draw(C--B, blue);

        markangle(n = 1, t, line(B, C), radius=20, red);
        markangle(n = 2, C, A, B, radius=20, red);

        dot("$A$", A, dir(210));
        dot("$B$", B, dir(90));
        dot("$C$", C, dir(5));
    \end{asy}
    \caption{Угол между касательной и хордой.}
    \label{fig:angle-between-tangent-and-chord}
\end{figure}


% \subsection{Касательные к окружности}
% Задачи на касающиеся объекты периодически попадются на олимпиадах. Могут касаться две окружности, а также прямая и окружность. Чтобы было удобнее обсуждать эту тему изучим еще одну новую технику -- счет в дугах.

\subsubsection{Счёт в дугах}

\begin{lemma}\label{lem:angle-between-chords}
    Две прямые $AB$ и $CD$ пересекаются внутри окружности в точке $P$, а точки $A$, $B$, $C$ и $D$ лежат на окружности. Тогда \[\angle APB = \frac{\overset{\frown}{AC} + \overset{\frown}{BD}}{2}.\]
    Здесь $AB$ и $CD$ -- это меньшие дуги окружности.
\end{lemma}

\begin{figure}[h]
    \centering
    \begin{asy}
        size(5cm);
        point O = (0, 0);
        circle omega = circle(O, 1); draw(omega);

        point A = angpoint(omega, 210);
        point C = angpoint(omega, 100);
        point B = angpoint(omega, 15);
        point D = angpoint(omega, 320);

        point P = intersectionpoint(line(A, B), line(C, D));


        draw(A--B, blue); draw(C--D, blue);
        draw(A--D, red+dashed);
        markangle(n = 3, C, P, A, radius=14, red);
        dot(P);
        markangle(n = 1, C, D, A, radius=14, red);
        markangle(n = 2, D, A, B, radius=14, red);
        dot("$A$", A, dir(210));
        dot("$B$", B, dir(15));
        dot("$C$", C, dir(100));
        dot("$D$", D, dir(320));
        dot("$P$", P, dir(50));
    \end{asy}
    \caption{Угол между хордами.}
    \label{fig:angle-between-chords}
\end{figure}

\begin{lemma}\label{lem:angle-between-secant}
    Две прямые $AB$ и $CD$ пересекаются вне окружности в точке $P$, а точки $A$, $B$, $C$ и $D$ лежат на окружности. Тогда \[\angle APB = \frac{\overset{\frown}{AC} - \overset{\frown}{BD}}{2}.\] Здесь $AB$ и $CD$ -- это меньшие дуги окружности. 
\end{lemma}
\begin{figure}[h]
    \centering
    \begin{asy}
        size(6cm, 5cm);
        point O = (0, 0);
        circle omega = circle(O, 1); draw(omega);

        point A = angpoint(omega, 210);
        point C = angpoint(omega, 90);
        point D = angpoint(omega, 5);
        point B = angpoint(omega, 320);

        point P = intersectionpoint(line(A, B), line(C, D));
        draw(A--P, blue); draw(C--P, blue);
        draw(A--D, red+dashed);
        markangle(n = 3, C, P, A, radius=14, red);
        dot(P);
        markangle(n = 1, C, D, A, radius=14, red);
        markangle(n = 2, B, A, D, radius=14, red);
        dot("$A$", A, dir(210));
        dot("$B$", B, dir(320));
        dot("$C$", C, dir(90));
        dot("$D$", D, dir(5));
        dot("$P$", P);
    \end{asy}
    \caption{Угол между секущими к окружности.}
    \label{fig:angle-between-secant}
\end{figure}

Эта техника и сама бывает стоящей полезной в несложных задачах. Сейчас мы с помощью нее и вычислим угол между касательной и хордой.

\begin{proposition}\label{prop:angle-between-tangent-and-chord}
    Пусть $AB$ -- хорда окружности, а $C$ -- точка касания касательной к окружности. Тогда угол между касательной и хордой равен вписанному углу, операющему на ту же дугу, что и хорда. То есть \[\angle ACB = \frac{\overset{\frown}{AB}}{2}.\]
\end{proposition}
\begin{proof}
    Пусть $P$ -- точка пересечения касательной с прямой $AB$. Тогда можем посчитать угол $\angle APC$ по \cref{lem:angle-between-secant}:
    \[
        \angle APC = \frac{\overset{\frown}{AC} - \overset{\frown}{BC}}{2}.
    \]

\end{proof}

\begin{figure}
    \centering
    \begin{asy}
        size(6cm, 5cm);
        point O = (0, 0);
        circle omega = circle(O, 1); draw(omega);

        point A = angpoint(omega, 210);
        point B = angpoint(omega, 70);
        point C = angpoint(omega, 5);

        line t = tangent(omega, C);
        point P = intersectionpoint(line(A, B), t);

        draw(t, red);
        draw(A--B);
        draw(C--A);
        draw(C--B, blue);
        draw(B--P, dashed);

        markangle(n = 1, t, line(B, C), radius=20, red);
        markangle(n = 2, C, A, B, radius=20, red);
        markangle(n = 3, A, P, C, radius=20, blue);

        dot("$A$", A, dir(210));
        dot("$B$", B, dir(90));
        dot("$C$", C, dir(5));
        dot("$P$", P, dir(40));
    \end{asy}
    \caption{Угол между касательной и хордой.}
    \label{fig:angle-between-tangent-and-chord}
\end{figure}


\section{Симметрии}

\section{Площади}
Давайте сейчас повторим то, чего мы уже знаем про площади различных многоугольников.

\begin{itemize}
    \item Равные многоулольники имеют равную площадь.
    \item Если многоугольник составлен из нескольких многоугольников, то его площадь равняется сумме площадей этих многоугольников.
    \item Площадь прямогоульника равняется прозведению двух его сторон.
    \item Площадь треугольника равняется половине произведения его основания на высоту.
    \item Площадь параллелограмма равна произведению его основания на высоту.
    \item Площадь ромба равна половине произведения его диагоналей
    \item Площадь трапеции равна произведению ее средней линии на высоту.
    \item Фигуры, имеющие равные площади, называются равновеликими. 
\end{itemize}

% TODO: нужно сделать картинки для всего-всего тут.

\begin{example}
    Докажите, что в треугольнике высоты обратно пропорциональна сторонам, к которым они проведены.
\end{example}

\begin{example}
    Медины $AA_1$ и $BB_1$ треугольника $ABC$ пересекаются в точке $M$. Найдите площадь треугольника $A_1MB_1$, если площадь треугольника $ABC$ равняется 1.
\end{example}

\begin{example}
    Докажите, что биссектриса треугольника делит его сторону на отрезки, пропорциональные двум другим сторонам.
\end{example}


\section{Ортоцентр треугольника}
Ортоцентр -- это такая особенная точка: конструкции, в которых используются его \textbf{симметрии} относительно чего-либо, \textbf{замечательно} связанны с описанной окружностью, и наоборот!

\noindent
\begin{minipage}{0.55\textwidth}
    \begin{theorem}\label{th:side-reflect}
        Если отразить ортоцентр относительно стороны, то он попадет на описанную окружность.
    \end{theorem}
\end{minipage}
\hspace{0.05\textwidth}
\begin{minipage}{0.4\textwidth}
    \begin{asy}
        size(5cm, 4cm);
        triangle t = triangleabc(11.1, 12.3, 11); draw(t, linewidth(bp));
        circle omega = circle(t); draw(omega);
        point H = orthocentercenter(t);

        point B_1 = intersectionpoints(altitude(t.VB), omega)[1]; 
        point A_1 = intersectionpoints(altitude(t.VA), omega)[1];

        point B_2 = foot(t.VB); point A_2 = foot(t.VA);

        draw(segment(H, B_1), dashed+red, StickIntervalMarker(2, 2, size=5));
        draw(segment(H, A_1), dashed+red, StickIntervalMarker(2, 1, size=5));

        perpendicularmark(t.AC, line(H, B_1), size=8, quarter=3, blue);
        perpendicularmark(t.BC, line(H, A_1), size=7, quarter=2, blue);
        
        dot(B_2); dot(A_2); dot(B_1); dot(A_1);
        dot(H, hpen);

        //draw(box((-2, 0), (14, 10.5)), invisible);
    \end{asy}
\end{minipage}\vspace{0.03\textwidth}
\begin{minipage}{0.55\textwidth}
    \begin{theorem}\label{th:middle-reflect}
        Если ортоцентр отразить относительно середины стороны, то он попадет на описанную окружность.
    \end{theorem}
\end{minipage}
\hspace{0.05\textwidth}
\begin{minipage}{0.4\textwidth}
    \begin{asy}
        size(5cm, 4cm);
        triangle t = triangleabc(9.6, 10.8, 12); draw(t, linewidth(bp));
        point H = orthocentercenter(t);
        
        circle Omega = circle(t); draw(Omega);
        point M_a = midpoint(t.CB);
    
        draw(segment(t.VC, t.VB), linewidth(bp), StickIntervalMarker(2, 1, size=7));
        
    
        point H_a = intersectionpoints(line(H, M_a), Omega)[0];

        dot(M_a); dot(H_a);
        draw(segment(H, H_a), red+dashed, StickIntervalMarker(2,2, size=7));
        dot(H, hpen);

        //draw(box((-1, -1), (13, 11)), invisible);
    \end{asy}
\end{minipage}\vspace{0.03\textwidth}
\begin{minipage}{0.55\textwidth}
    \begin{corollary}\label{cor:diametr}
        Точка из теоремы \ref{th:middle reflect} диаметрально противоположна противолежащей стороне вершине.
    \end{corollary}
\end{minipage}
\hspace{0.05\textwidth}
\begin{minipage}{0.4\textwidth}
    \begin{asy}
        size(5cm, 4cm);
        triangle t = triangleabc(11.8, 14, 11); draw(t, linewidth(bp)); 
        point H = orthocentercenter(t);
        point O = circumcenter(t);
        
        circle Omega = circle(t); draw(Omega);
        point M_a = midpoint(t.CB);
    
        draw(segment(t.VC, t.VB), linewidth(bp), StickIntervalMarker(2, 1, size=7));
        
        point H_a = intersectionpoints(line(H, M_a), Omega)[1];
        draw(segment(H, H_a), gray, StickIntervalMarker(2,2, size=7));
        draw(segment(t.VA, H_a), red);

        dot(H_a); dot(M_a);
        dot(H, hpen);
        dot(O, filltype=FillDraw(fillpen=white, drawpen=black));

        //draw(box((-2, -1), (13, 12)), invisible);
    \end{asy}
\end{minipage}\vspace{0.03\textwidth}
\begin{minipage}{0.55\textwidth}
    \begin{corollary}\label{cor:distance from O}
        Расстояние от вершины треугольника до ортоцентра в $2$ раза больше расстояния от центра описанной окружности до противолежащей стороны.         
    \end{corollary}
\end{minipage}
\hspace{0.05\textwidth}
\begin{minipage}{0.4\textwidth}
    \begin{asy}
        size(5cm, 4cm);
        triangle t = triangleabc(9.5, 12.5, 12.2); draw(t, linewidth(bp));
        circle omega = circle(t); 
        point H = orthocentercenter(t);
        point O = circumcenter(t);
        point M = midpoint(t.AB);

        draw(segment(t.VA, t.VB), StickIntervalMarker(2, 2, size=5));
        perpendicularmark(t.AB, line(O, M), blue, size=7);

        draw(segment(O, M), gray, StickIntervalMarker(1, 1, size=5)); 
        draw(segment(t.VC, H), red, StickIntervalMarker(2, 1, size=5));
        dot(midpoint(segment(t.VC, H)));

        
        dot(H, hpen);
        dot(O, filltype=FillDraw(fillpen=white, drawpen=black));
    \end{asy}
\end{minipage}\vspace{0.03\textwidth}
\begin{minipage}{0.55\textwidth}
    \begin{lemma}[Окружность Джонсона]\label{lem:(ABH)}
        $(ABC) = (ABH)$, т.е. окружности, описанные вокруг $\triangle ABC$ и $\triangle ABH$ равны.
    \end{lemma}
\end{minipage}
\hspace{0.05\textwidth}
\begin{minipage}{0.4\textwidth}
    \begin{asy}
        size(5cm, 4cm);
        triangle t = triangleabc(9, 11, 10); draw(t, linewidth(bp)); 
        circle abc = circle(t); draw(abc);
        point H = orthocentercenter(t);
    
        draw(circle(point(t.VC), point(t.VB), H), dashed+red);
        dot(H, hpen);
    \end{asy}
\end{minipage}\vspace{0.03\textwidth}
\begin{minipage}{0.55\textwidth}
    \begin{definition}[Изогональное сопряжение\setcounter{footnote}{0}\footnotemark]\label{def:isogonal}
        Точки $P$, $Q$ называются изогонально сопряженными, если $\angle PAB = \angle QAC$, $\angle PBC = \angle QBA$, $\angle PCB = \angle QCA$.
    \end{definition}
    \begin{theorem}\label{th:OHisogonal}
        Ортоцентр и центр описанной окружности изогонально сопряжены.
    \end{theorem}
\end{minipage}
\footnotetext{Можно думать об изогональном сопряжении, как о симметрии относительно биссектрисы.}
\hspace{0.05\textwidth}
\begin{minipage}{0.4\textwidth}
    \begin{asy}
        size(5cm, 4cm);
        triangle t = triangleabc(11, 9, 12); draw(t, linewidth(bp));
        point H = orthocentercenter(t);
        point O = circumcenter(t);

        draw(segment(t.VA, H));
        draw(segment(t.VA, O));

        draw(segment(t.VC, H));
        draw(segment(t.VC, O));

        markangle(n = 1, line(t.AB), line(t.VA, O), radius=35, red);
        markangle(n = 1, line(t.VA, H), line(t.AC), radius=30, red+dashed);

        markangle(n = 2, line(t.VC, O), line(t.BC), radius=30, dashed+blue);
        markangle(n = 2, rotate(180)*line(t.AC), line(t.VC, H), radius=30, blue);

        dot(H, hpen);
        dot(O, filltype=FillDraw(fillpen=white, drawpen=black));
    \end{asy}
\end{minipage}

\begin{minipage}{0.55\linewidth}
    \begin{definition}\label{def:incenter}
        Инцетр -- это центр, вписанной в многоугольник окружности.
    \end{definition}
    \begin{definition}\label{def:orthotriangle}
        Ортотреугольник -- это треугольник, вершины которого являются основаниями высот исходного треугольник.
    \end{definition}
    \begin{lemma}\label{lem:H -- incenter orthotriangle}
        Ортоцентр является инцентром для ортотреугольника.
    \end{lemma}
\end{minipage}
\hspace{0.05\linewidth}
\begin{minipage}{0.4\linewidth}
    \begin{asy}
        size(4.9cm, 4cm);
        triangle t = triangleAbc(60, 3, 4); draw(t, linewidth(bp));
        point H = orthocentercenter(t);

        triangle t_1 = pedal(t, H); draw(t_1);

        draw(segment(t.VA, t_1.VA), grey);
        draw(segment(t.VB, t_1.VB), grey);
        draw(segment(t.VC, t_1.VC), grey);

        perpendicularmark(t.AB, line(t.VC, t_1.VC), size=7, blue);
        perpendicularmark(t.CB, line(t.VA, t_1.VA), size=7, blue, quarter=3);
        perpendicularmark(t.AC, line(t.VB, t_1.VB), size=7, blue, quarter=4);

        markangle(n = 1, point(t.VB), point(t_1.VC), point(t_1.VA), radius=12, red);
        markangle(n = 1, point(t_1.VB), point(t_1.VC), point(t.VA), radius=10, red);

        markangle(n = 2, point(t.VA), point(t_1.VB), point(t_1.VC), radius=8, red);
        markangle(n = 2, point(t_1.VA), point(t_1.VB), point(t.VC), radius=10, red);

        markangle(n = 3, point(t.VC), point(t_1.VA), point(t_1.VB), radius=6, red);
        markangle(n = 3, point(t_1.VC), point(t_1.VA), point(t.VB), radius=4, red);
    \end{asy}
\end{minipage}\vspace{0.03\linewidth}
\begin{minipage}{0.55\linewidth}
    \begin{corollary}\label{cor:AH perp B1C1}
        Радиусы описанной окружности, проведённые к вершинам треугольника, перпендикулярны соответствующим сторонам ортотреугольника.
    \end{corollary}
\end{minipage}
\hspace{0.05\linewidth}
\begin{minipage}{0.4\linewidth}
    \begin{asy}
        size(4.9cm, 4cm);
        triangle t = triangleAbc(60, 6, 7.3); draw(t, linewidth(bp));
        point O = circumcenter(t);
        
        point H_a = foot(t.VA); draw(segment(t.VA, H_a), grey);
        point H_b = foot(t.VB); draw(segment(t.VB, H_b), grey);
        draw(segment(H_a, H_b));
        perpendicularmark(t.CB, line(t.VA, H_a), blue, size=7, quarter=4);
        perpendicularmark(t.AC, line(t.VB, H_b), blue, size=7, quarter=3);

        draw(segment(t.VC, O), red);
        perpendicularmark(line(t.VC, O), line(H_a, H_b), dashed+blue, size=7, quarter=3);
        
        dot(O, filltype=FillDraw(fillpen=white, drawpen=black));
    \end{asy}
\end{minipage}\vspace{0.03\linewidth}
\begin{minipage}{0.55\linewidth}
    \begin{lemma}\label{lem:4R^2}
        Сумма квадратов расстояния от вершины треугольника до ортоцентра и длины стороны, противолежащей этой вершине, равна квадрату диаметра описанной окружности.
    \end{lemma}
\end{minipage}
\hspace{0.05\linewidth}
\begin{minipage}{0.4\linewidth}
    \begin{asy}
        size(4.9cm, 4cm);
        triangle t = triangleAbc(70, 7, 11); draw(t, linewidth(bp)); label(t);
        point H = orthocentercenter(t);
        point O = circumcenter(t);

        draw(segment(t.VA, H));
        draw(segment(t.VC, O), red);

        dot(H, hpen);
        dot(O, filltype=FillDraw(fillpen=white, drawpen=black));
    \end{asy}
    \vspace{-0.3cm}
    $$AH^2 + BC^2 = 4 \cdot OC^2$$
\end{minipage}\vspace{0.03\linewidth}
\begin{minipage}{0.55\linewidth}
    \begin{lemma}\label{lem:cos}
        Если $AA_1$ и $BB_1$ -- высоты треугольника $ABC$, то $\triangle ABC \sim \triangle A_1B_1C, \quad k = \cos \angle C$.
    \end{lemma}
\end{minipage}
\hspace{0.05\linewidth}
\begin{minipage}{0.4\linewidth}
    \begin{asy}
        size(4.9cm, 4cm);
        triangle t = triangleabc(6, 5, 6.2); draw(t, linewidth(bp)); label(t); 
        point A_1 = foot(t.VA); point B_1 = foot(t.VB);
    
        draw(segment(t.VA, A_1), grey); perpendicularmark(t.BC, line(t.VA, A_1), blue, size=7);
        draw(segment(t.VB, B_1), grey); perpendicularmark(t.AC, line(t.VB, B_1), blue, size=7, quarter=3);

        draw(segment(A_1, B_1), red);
        
        dot("$A_1$", A_1, E+0.5N); dot("$B_1$", B_1, W+0.5N);
    \end{asy}
    \iffalse
    \begin{equation*}
        \begin{split}
            &\triangle ABC \sim \triangle A_1B_1C \\
            &k = \frac{AC}{A_1C_1} = cos \angle C  
        \end{split}
    \end{equation*}
    \fi
\end{minipage}

\subsection{Другие популярные уголки}
\begin{lemma}\label{lem:con-or-kite}
    Если в четырехугольнике $ABCD$, $AC$ -- биссектриса угла $A$ и $BC = CD$, то этот четырехугольник является либо вписанным, либо дельтойдом\footnote{\emph{Дельтойдом (кайтом)} называется четырехугольик, у которого есть две пары равных смежных сторон}.
\end{lemma}

\begin{figure}[h]
    \centering
    \begin{asy}
        size(6cm, 5cm);
        triangle t = triangleabc(13, 9, 10);

        circle omega = circle(t); 
        point M = intersectionpoints(omega, bisector(t.VC))[0];

        draw(point(t.VA)--point(t.VC)--point(t.VB)--M--cycle, linewidth(bp));

        draw(point(t.VC)--M, blue);
        markangle(n = 1, point(t.VA), point(t.VC), M, radius=16, red);
        markangle(n = 1, M, point(t.VC), point(t.VB), radius=18, red);

        draw(point(t.VA)--M, StickIntervalMarker(1, 1, size=7));
        draw(point(t.VB)--M, StickIntervalMarker(1, 1, size=7));

        point H = intersectionpoint(t.BC, perpendicular(t.VA, line(t.VC, M)));
        draw(M--H, dashed+red, StickIntervalMarker(1, 1, size=7));
        dot(H);

        draw(omega, red+dashed);

        label("$A$", point(t.VC), N);
        label("$B$", point(t.VB), E);
        label("$C$", M, S);
        label("$D$", point(t.VA), W);
    \end{asy}
    \label{fig:con-or-kite}
    \caption{Случай вписанности и дельтойда.}
\end{figure}


\section{Подобие треугольников} 

% придумать много простых


\section{Степень точки}
\begin{definition}[Степень точки]\label{def:pow}
    Степень точки $P$, находящейся на расстоянии $d$ от центра окружности $\omega$ радиусом $r$, относительно этой же окружности: $$\pow(P, \omega) = d^2-r^2.$$
\end{definition}

\begin{theorem}\label{th:tan}
    Если прямая $\ell \ni P$ касается окружность в точке $K$, то $$\pow(P, \omega) = PK^2.$$
\end{theorem}

\begin{theorem}\label{th:pow}
    Если прямая $\ell \ni P$ пересекает окружность $\omega$ в точках $A$ и $B$, тогда $$\pow(P, \omega) = \overrightarrow{PA} \cdot \overrightarrow{PB}.$$
\end{theorem}

\begin{figure}[ht]
    \centering
    \begin{asy}
        size(12cm, 6cm);
        point O = (2, 0);
        circle omega = circle(O, 1.7); draw(omega);
        point P = (-2.5, 2); dot("$P$", P, dir(70));

        point A_2 = angpoint(omega, 40); dot("$A_2$", A_2, dir(40));
        point A_1 = intersectionpoints(omega, line(P, A_2))[1]; dot("$A_1$", A_1, dir(100));
        draw(line(P, A_2));

        point B_2 = angpoint(omega, -80); dot("$B_2$", B_2, dir(-100));
        point B_1 = intersectionpoints(omega, line(P, B_2))[1]; dot("$B_1$", B_1, dir(-140));
        draw(line(P, B_2));

        draw(B_1--A_2, red+dashed); draw(B_2--A_1, red+dashed);

        markangle(n = 1, P, A_2, B_1, radius=25, blue);
        markangle(n = 1, A_1, B_2, P, radius=25, blue);

        markangle(n = 2, B_2, P, A_2, radius=20, blue+bp);

        draw(box((-3,-2), (4, 2.3)), invisible);
    \end{asy}
\end{figure}

\begin{corollary}[Теорема о касательной и секущей]\label{cor:tangent_and_sector}
    Если из точки $P$, проведена касательная $PK$ к окружности $\omega$ и прямая $(\ell \ni P)$ пересекает окружность $\omega$ в точках $A$ и $B$, тогда $$PK^2 = PA \cdot PB.$$
\end{corollary}

\begin{theorem}[Главная теорема о степени точки]\label{th:superpow}
    Если через точку $P$ проходят две прямые, которые пересекают окружность $\omega$ в точках $A_1, A_2$ и $B_1, B_2$ соответственно, то $$\pow(P, \omega) = \overrightarrow{PA_1}\cdot\overrightarrow{PA_2} = \overrightarrow{PB_1}\cdot\overrightarrow{PB_2}.$$
\end{theorem}


\subsection{Радикальная ось}
\begin{theorem}\label{th:radaxis}
    Геометрическое место точек \emph{(ГМТ)}, степени которых относительно двух неконцентрических окружностей равны, есть прямая, перпендикулярная линии центров этих окружностей.
\end{theorem}

\begin{definition}[Радикальная ось]\label{def:radaxis}
    Прямая, состоящая из точек, степени которых относительно двух данных окружностей равны, называется радикальной осью этих окружностей.
\end{definition}

\begin{figure}[ht]
    \centering
    \begin{asy}
        size(12cm, 6cm);
        point O_1 = (0, 0); point O_2 = (4, 0);
        circle omega_1 = circle(O_1, 1.5); circle omega_2 = circle(O_2, 1);
        draw(omega_1); draw(omega_2);

        line ra = radicalline(omega_2, omega_1);
        point P = point(ra, 0.9);

        line l_1 = tangents(omega_1, P)[1];
        point T_1 = intersectionpoints(l_1, omega_1)[0]; dot(T_1);

        line l_2 = tangents(omega_2, P)[0];
        point T_2 = intersectionpoints(l_2, omega_2)[0]; dot(T_2);
        
        draw(ra, red);
        draw(line(O_1, O_2), gray);

        draw(P--T_1, dashed+blue, StickIntervalMarker(1, 1, size=8));
        draw(P--T_2, dashed+blue, StickIntervalMarker(1, 1, size=8));

        dot("$O_1$", O_1, filltype=FillDraw(fillpen=white, drawpen=black), dir(120)); dot("$O_2$", O_2, filltype=FillDraw(fillpen=white, drawpen=black), dir(60));

        dot(P);
        perpendicularmark(line(O_1, O_2), ra, blue+dashed, size=8, quarter=1);

        draw(box((-2, -2), (5, 2)), invisible);
    \end{asy}
    \caption{Радикальная ось двух окружностей.}
\end{figure}

\begin{theorem}[Радикальный центр]\label{th:radcenter}
    Радикальные оси трех окружностей либо конкурентны, либо параллельны.
\end{theorem}
\begin{figure}
    \centering
        \begin{asy}
        size(5.5cm);
        point O_1 = (-1, 0); point O_2 = (3, 0); point O_3 = (0, 2.5);
        circle omega_1 = circle(O_1, 1.5); circle omega_2 = circle(O_2, 1); circle omega_3 = circle(O_3, 1.6);
        draw(omega_1, blue); draw(omega_2, blue); draw(omega_3, blue);

        line O1O2 = radicalline(omega_1, omega_2); draw(O1O2, red+dashed);
        line O1O3 = radicalline(omega_1, omega_3); draw(O1O3, red);
        line O2O3 = radicalline(omega_3, omega_2); draw(O2O3, red);


        dot(radicalcenter(omega_1, omega_2, omega_3), blue+4);
        
        draw(box((-3.7, -2), (5, 4.4)), invisible);
    \end{asy}
    \qquad\qquad   
    \begin{asy}
        size(5.5cm);
        point O_1 = (-5, 0); point O_2 = (3, 0); point O_3 = (12, 0);
        circle omega_1 = circle(O_1, 3); circle omega_2 = circle(O_2, 2); circle omega_3 = circle(O_3, 4);
        draw(omega_1, blue); draw(omega_2, blue); draw(omega_3, blue);

        line O1O2 = radicalline(omega_1, omega_2); draw(O1O2, red);
        line O1O3 = radicalline(omega_1, omega_3); draw(O1O3, red+dashed);
        line O2O3 = radicalline(omega_3, omega_2); draw(O2O3, red);

        draw(box((0, -10), (2, 8)), invisible);
    \end{asy}
    \caption{Радикальный центр трех окружностей.}
\end{figure}

\newpage
\begin{theorem}\label{th:deltapow}
    $AC \perp BD$, если $$\pow(B, \omega_a) - \pow(B, \omega_c) = \pow(D, \omega_a) - \pow(D, \omega_c)$$
\end{theorem}
\begin{figure}[ht]
    \centering
    \begin{asy}
        size(12cm, 6cm);
        point A = (-7, 0); point C = (15, 0);
        circle omega_a = circle(A, 4); circle omega_c = circle(C, 6);
        draw(Label("$\omega_a$", Relative(0.375)), omega_a, red+dashed+0.8*bp); draw(Label("$\omega_c$", Relative(0.1)), omega_c, blue+dashed+0.8*bp);

        point B = (3, 11); point D = (3, -5);
        draw(line(A, C), gray);
        draw(line(B, D), gray);

        draw(A--B--C--D--A);

        perpendicularmark(line(A, C), line(B, D), blue, size=10);

        dot("$A$", A, dir(225)); dot("$C$", C, dir(315));
        dot("$B$", B, dir(30)); dot("$D$", D, dir(330)); 
    \end{asy}
\end{figure}


\newpage
\renewcommand{\thesubsection}{\roman{subsection}}
\setcounter{subsection}{0}

\section*{Задачи}
\markboth{Задачи}{}
\addcontentsline{toc}{section}{Задачи}
% TODO: не забыть вставить задачки из симплов.
% TODO: вставить и посмотреть задачи из 2.10 Гордина.
% TODO: можно брать задачи из разминок всяких каналов!
\subsection{Счёт углов-I}
\begin{tasks}
    \item Биссектриса угла $A$ треугольника $ABC$ пересекает его описанную окружность в точке $L$. Докажите, что $BL = CL$.    % точно ли оставить?

    \item Биссектрисы треугольника $ABC$ пересекают описанную окружность $(ABC)$ в точках $A_1, B_1, C_1$. Докажите, что высоты треугольника $A_1B_1C_1$ лежат на прямых $AA_1, BB_1, CC_1$. % ну тут тоже не на вписанные..

    \item Точки $A, B, C, D$ лежат на окружности. Точки $M , N , K, L$ --- середины дуг $AB$, $BC$, $CD$, $DA$ соответственно. Докажите, что $M K \perp N L$.

    \item \named{Лемма Фусса}\label{lem:fuss}{Окружности $\omega_1$ и $\omega_2$ пересекаются в точках $A$ и $B$. Через точку $A$ проведена прямая вторично пересекающая окружность $\omega_1$ в точке $A_1$ и окружность $\omega_2$ в точке $A_2$. Точки $B_1$ и $B_2$ для прямой через точку $B$ определяются аналогично. Докажите, что $A_1B_1 \parallel A_2B_2$.}   
    
    \solution{
        По \cref{lem:concycle} $\angle B_1A_1A = \angle ABB_2 = 180^\circ - \angle B_2A_2A \Rightarrow \angle B_1A_1A_2 + \angle B_2A_2A_1 = 180^\circ \Rightarrow A_1B_1 \parallel A_2B_2.$
    }

    % \item В равнобедренном треугольник $ABC$ $(AB=AC)$ на меньшей дуге $AB$ окружности $(ABC)$ взята точка $D$. На продолжении отрезка $AD$ за точку $D$ выбрана точка $E$ так, что точки $A$ и $E$ лежат по одну сторону относительно прямой $BC$. Окружность $(BDE)$ пересекает прямую $AB$ в точке $F$. Докажите, что $EF \parallel BC$.

    % 
    % \solution{
    %     По \cref{lem:fuss}  $E$ и $F$ --- вторые точки пересечения окружности $(BDE)$ с прямыми $AD$ и $AB$ соответственно. Тогда прямая $EF$ параллельна касательной к $(ABC)$ в точке $A$. И уже эта касательная параллельна $BC$, тогда и $EF$ тоже.
    % }

    \item В трапеции $ABCD$ проведена окружность $\omega$, проходящая через точки $A$ и $D$. Окружность пересекает боковые стороны $AB$ и $CD$ (или их продолжения) в точках $N$ и $M$ соответственно. Докажите, что если точка пересечения прямых $BM$ и $CN$ равноудалена от точек $A$ и $D$, то она лежит на окружности $\omega$. 

    \solution{
        $AD \parallel BC$, тогда по обратной \cref{lem:fuss} $NBCM$ --- вписанный. Тогда $\angle BNC = \angle BMC$. \

        По обратной \cref{lem:concycle} для четырехугольников $ANPD$ и $APMD$ $\angle BNC = \angle PDA$ и $\angle BMC = \angle PAD$. Отсюда следует, что треугольник $APD$ --- равнобедренный, а значит $P$ равноудалена от $A$ и $D$.
    }
    
    \item В остроугольном треугольнике $ABC$ на высоте, проведённой из вершины $A$, выбрана точка $P$. Пусть $B_1$ и $C_1$ --- проекции точки $P$ на прямые $AC$ и $AB$ соответственно. 
    \begin{tasks}
        \item Докажите, что точки $B$, $C$, $B_1$, $C_1$ концикличны. \label{lem:projections}
        
        \solution{
            Пусть точка $D$ --- основания высоты из вершины $A$. Тогда $BDPC_1$ и $AC_1PB_1$ --- вписаные четырехугольники. По \cref{lem:concycle} $\angle ABC = \angle APC_1$ и $\angle APC_1 = \angle AB_1C_1$. Тогда по обратной \cref{lem:concycle} $BCC_1B_1$ --- вписанный четырехугольник.
        }
        
        \moditem{*} Докажите, что отрезок, соединяющий проекции точек $B_1$ и $C_1$, на прямые $AB$ и $AC$ соответственно, параллелен стороне $BC$. \label{lem:3b}
        
        \solution{
            По \cref{lem:projections} $BCC_1B_1$ --- вписанный, а также $B_1C_1C_2B_2$ ($B_1C_1$ --- диаметр). Тогда по \cref{lem:concycle} $\angle ABC = \angle AB_1C_1 = \angle AC_2B_2 \Rightarrow B_2C_2 \parallel BC$.
        }
    \end{tasks}

    % \item В остроугольном треугольнике $ABC$ проведена высота $AD$. Пусть точки $K$ и $L$ --- проекции точки $D$ на стороны $AB$ и $AC$ соответственно. Известно, что $\angle BAC = 72^\circ, \angle ABL = 30^\circ$. Чему равен угол $\angle DKC$?
    % 
    % \solution{
    %     По \cref{lem:projections} $BCLK$ --- вписанный, тогда $\angle ABL =\angle LCK$.
    % 
    %     $\angle DKC = \angle BDK - \angle DCK$. $\angle BDK = \angle BAD$ (углы при высоте прямоугольного треугольника). 
    %     
    %     $\angle DCK = \angle ACD - \angle LCK = 90^\circ - \angle CAD - \angle LCK = 90^\circ - \angle CAD - \angle ABL$.
    % 
    %     $\angle DKC = \angle BAD - 90^\circ + \angle CAD + \angle ABL = \angle BAC + \angle ABL - 90^\circ = 72^\circ + 30^\circ - 90^\circ = 12^\circ$.
    % }

    % \moditem{*} \named{Окружность Тейлора}{Докажите, что шесть точек в виде шести проекций трёх оснований высот треугольника, пересекающих каждую сторону, на две оставшиеся стороны лежат на одной окружности.}

    % \solution{
    %     Пусть точки $H_a$, $H_b$ и $H_c$ --- основания высот из соответствующих вершин треугольника $ABC$. Пусть $B_a$ и $C_a$ --- проекции точки $H_a$ на прямые $AB$ и $AC$ соответственно. Точки $A_b$, $C_b$, $A_c$ и $B_c$ определяются аналогично. 
    %     
    %     Тогда по \cref{lem:projections} $BCB_AC_A$ --- вписанный. Тогда по \cref{lem:concycle} $\angle ACB = \angle AC_aB_a$.
    %     
    %     По \cref{lem:3b} $AB \parallel A_bB_a \Rightarrow \angle AC_aB_a = \angle A_bB_aC_a$, и $AC \parallel A_cC_a \Rightarrow \angle ACB = \angle A_cC_aB$.
    % 
    %     Тогда по обратной \cref{lem:concycle} $A_cA_bB_aC_a$ --- вписанный. Аналогично $B_aB_cC_bA_b$ и $C_bC_aA_cB_c$ --- вписанные. Тогда и $A_cA_bB_aB_cC_bC_a$ --- вписанный, т.к. точки лежат на сторонах треугольника (строго позже).
    % }

    \item \begin{tasks}
        \item \named{Точка Микеля треугольника}{На сторонах $AB$, $BC$ и $AC$ треугольника $ABC$ или их продолжениях, выбраны точки $C_1$, $B_1$ и $A_1$ соответственно. Докажите, что окружности $(AB_1C_1)$, $(A_1BC_1)$ и $(A_1B_1C)$ пересекаются в одной точке.}  \label{th:miquel's theorem}
        
        \solution{
            Пусть $(AB_1C_1) \cap (A_1BC_1) = P$. Будем доказывать, что $P \in (A_1B_1C)$. По \cref{lem:concycle} $\angle BC_1P = \angle CA_1P = \angle AB_1P$. Отсюда по обратной \cref{lem:concycle} точки $A_1$, $B_1$, $C$ и $P$ концикличны.
        }
        
\moditem{*} \named{Точка Микеля четырехсторонника}{Прямая $\ell$ пересекает прямые содержащие стороны треугольника $AB$, $BC$ и $AC$ в точках $F$, $D$, $E$ соотвественно. Тогда окружности $4$ окружности $(ABC)$, $(AFE)$, $(BFD)$ и $(CDE)$ имеют общую точку.}  \label{th:miquel's point}

        \solution{
            Пусть на первой прямой лежат точки $A$, $F$ и $B$, на второй $B$, $D$ и $C$, на третьей $C$, $A$ и $E$ и на четвертой $E$, $D$ и $F$. Тогда по \cref{th:miquel's theorem} для $\triangle ABC$ и точек $F$, $D$ и $E$
            \begin{equation}
                (AFE) \cap (BFD) \cap (CDE) = M. \label{eq:th:miquel's point 1}
            \end{equation}
    
            По \cref{th:miquel's theorem} для $\triangle AFE$ и точек $B$, $D$ и $C$
            \begin{equation}
                (ABC) \cap (FBD) \cap (EDC) = G. \label{eq:th:miquel's point 2}
            \end{equation}
            
            Но по \cref{eq:th:miquel's point 1,eq:th:miquel's point 2} $G \equiv M$. Отсюда следует, что все нужные окружности пересекаются в одной точке.
        }
    \end{tasks}

    
    \item В треугольнике $ABC$ точки $B_1$ и $C_1$ --- основания высот, проведенных из вершин $B$ и $C$ соответственно. Точка $D$ --- проекция точки $B_1$ на сторону $AB$, точка $E$ --- пересечения перпендикуляра, опущенного из точки $D$ на сторону $BC$, с отрезком $BB_1$. Докажите, что $EC_1 \perp BB_1$. 

    \solution{
        Нужно доказать, что $DC_1EB_1$ --- вписанный, тогда утверждение верно. $B_1EFC$ --- вписанный, тогда по \cref{lem:concycle} $\angle B_1CF = \angle B_1ED$. Также $BCC_1B_1$ --- вписанный, тогда, опять же, по \cref{lem:concycle} $\angle BCB_1 = \angle B_1C_1D$. Тогда, раз $\angle B_1ED = \angle B_1ED = \angle B_1C_1D$, то $DC_1EB_1$ --- вписанный.
    }

    \item На гипотенузе $AC$ прямоугольного треугольника $ABC$ во вне\-шнюю сторону построен квадрат с центром в точке $O$. Докажите, что $BO$ --- биссектриса угла $ABC$. 

    \solution{
        $ABCO$ --- вписанный, т.к. $\angle B = \angle O = 90^\circ$. $AO = OC$, т.к. это половины диагоналей квадрата. Тогда $BO$ --- биссектриса угла $ABC$.
    }

    % TODO: Эту задачу нужно будет в сложные уголки
    \item В треугольнике $ABC$ угол $A$ равен $60^\circ$. Биссектрисы треугольника $BB_1$ и $CC_1$ пересекаются в точке $I$. Докажите, что $IB_1=IC_1$. 

    \solution{
        \begin{lemma}\label{lem:Iangle}
            Если в треугольнике $ABC$, точка $I$ --- инцентр, то $$\angle AIC = 90^\circ + \frac{1}{2}\angle ABC$$
        \end{lemma}

        По \cref{lem:Iangle} $\angle BIC = 90^\circ + \frac 1 2 \angle BAC = 90^\circ + 30^\circ = 120 ^\circ.$ Тогда $AB_1IC_1$ --- вписанный. $AI$ --- биссектриса, поэтому $IB_1 = IC_1.$
    }

        \item Прямая $\ell$ касается описанной окружности треугольника $ABC$ в точке $B$. Точки $A_1$ и $C_1$ --- проекции точки $P \in \ell$ на прямые $AB$ и $BC$ соответственно. Докажите, что $A_1C_1 \perp AC$. 

    \solution{
        \begin{lemma}
            Угол между касательной и хордой окружности, равен половине градусной меры дуги, стягиваемой данной хордой.
        \end{lemma}
        \begin{corollary}\label{cor:tangentangle}
            Если к окружности $(ABC)$ провели касательную $BK$, то: $\angle BAC = \angle CBK$.
        \end{corollary}

        По \cref{cor:tangentangle} $\angle PBA_1 = \angle BAC$. $PC_1BA_1$ --- вписанный, поэтому $\angle PC_1A_1 = \angle PBA_1$.

        $\angle PC_1A_1 + \angle A_1C_1B = 90^\circ = \angle BAC + \angle(AB, A_1C_1) \Rightarrow AC \perp A_1C_1.$
    }

    \item Продолжения противоположных сторон $AB$ и $CD$ вписанного четырехугольника $ABCD$ пересекаются в точке $M$, а сторон $AD$ и $BD$ --- в точке $N$. Докажите, что биссектрисы углов $AMD$ и $DNC$ взаимно перпендикулярны.

    \item Прямая проходяшая через точку $A$ и центр $O$ описанной окружности треугольника $ABC$, вторично пересекает описанную окружность в точке $N$. Докажите, что треугольники $BON$ и $CON$ равнобедренные.

    \moditem{*} Окружности $\omega_1$ и $\omega_2$ пересекаются в точках $A$ и $B$. Прямая $\ell$ касается окружностей $\omega_1$ и $\omega_2$ в точках $P$ и $Q$ соответственно (точка $B$\footnote{Точка $B$ называется точкой Шалтая треугольника $APQ$.} лежит внутри треугольника $APQ$). Прямая $BP$ вторично пересекает $\omega_2$ в точке $T$. Докажите, что $AQ$ --- биссектриса угла $\angle PAT$. 

    \solution{
        По \cref{cor:tangentangle} для прямой $PQ$ и окружностей $\omega_1$ и $\omega_2$ $\angle BPQ = \angle BAP$ и $\angle BQP = \angle BAQ$. Тогда угол $TBQ = \angle BAQ + \angle BAP = \angle PAQ$ (внешний в треугольнике $BPQ$).

        Так как $BQTA$ --- вписанный, то $\angle TBQ = \angle TAQ = \angle PAQ$. Тогда и получается, что $AQ$ --- биссектриса угла $PAT$.
    }

    % \moditem{*} Пусть $AA_1$, $BB_1$ и  $CC_1$ --- высоты остроугольного треугольника $ABC$. Докажите, что проекции точки $A_1$ на прямые $AB$, $AC$, $BB_1$, $CC_1$ коллинеарны. 

    % \solution{
    %     Докажем, что проекции на $AB$, $BB_1$ и $CC_1$ коллинеарны. Аналогично будет следовать, что и проекция на $AC$ лежит на этой прямой. Пусть $X$, $Y$ и $Z$ --- проекции на $AB$, $BB_1$ и $CC_1$ соответственно, а $H$ --- ортоцентр. Тогда по \cref{lem:projections,lem:concycle} $\angle BCH = \angle HYZ$. $CH \parallel A_1X$, поэтому $\angle BCH = \angle BA_1X.$ Т.к. $BXYA_1$ --- вписанный, то $\angle BA_1X = \angle BYX$. Получили, что $\angle BYX = \angle HYZ$ --- вертикальные углы, значит $Y \in XZ$.
    % 
    % }
    % 
    % \moditem{*} В треугольнике $ABC$ точки $D$ и $E$ --- основания биссектрис из углов $A$ и $C$ соответственно, а точка $I$ --- центр вписанной в треугольник $ABC$ окружности. Точки $P$ и $Q$ --- пересечения прямой $DE$ с $(AIE)$ и $(CID)$ соответственно, причем $P \neq E, Q \neq D$. Докажите, что $\angle EIP = \angle DIQ$.\footnote{Сделав инверсию в точке $A$ или $I$, получите задачу с Высшей Пробы 2024.}

    % \solution{
    %     Т.к. $AEPI$ и $CQDI$ --- вписанные, то $\angle PIE = \angle PAE$ и $\angle DIQ = \angle DCQ$. Пусть точка $T$ --- пересечение прямых $AP$ и $CQ$. Тогда нужно доказывать, что $APTC$ --- вписанный.

    %     Пусть $\angle ABC = 2\beta$, тогда по \cref{lem:Iangle} $\angle AIC = 90^\circ + \frac 1 2 \angle ABC = 90^\circ + \beta$, тогда внешние углы $PIA$ и $DIA$ равны $90^\circ-\beta$. 

    %     По \cref{lem:concycle} для четырехугольников $AEPI$ и $CQDI$ $\angle PIA = \angle TPQ$ и $\angle DIA = TQP$. Тогда $\angle PTQ = 180^\circ - 2(90^\circ-\beta) = 2\beta = \angle ABC$. Тогда $APTC$ --- вписанный.
    % }
\end{tasks}

\subsection{Симметрия}  % TODO: взять из параграфа 2.6 Гордина
% \begin{tasks}
    \item Пусть $M$ и $N$  --- середины оснований трапеции. Докажите, что если прямая $MN$ перпендикулярна основаниям, то трапеция равнобедренная.
    \item Пусть $M$ --- середина отрезка $AB$. Точки $A'$, $B'$ и $M'$ --- образы точек соответственно $A$, $B$ и $M$ при симметрии относительно некоторой точки $O$. Докажите, что $M'$ --- середина $A'B'$.
    \item На противоположных сторонах параллелограмма как на сторонах построены вне параллелограмма два квадрата. Докажите, что прямая, соединяющая их центры, проходит через центр параллелограмма.
    \item Докажите, что точки, симметричные произвольной точке относительно середин сторон квадрата, являются вершинами некоторого квадрата.
    \item Даны две концентрические окружности $S_1$ и $S_2$. Постройте прямую, на которой эти окружности высекают три равных отрезка.
    \item Противоположные стороны выпуклого шестиугольника попарно равны и параллельны. Докажите, что он имеет центр симметрии.
    \item Диагонали $AC$ и $BD$ параллелограмма $ABCD$ пересекаются в точке $O$. Докажите, что окружности, описанные около треугольников $AOB$ и $COD$, касаются
    \item Фигура имеет две перпендикулярные оси симметрии. Докажите, что она имеет центр симметрии.
    \item Точки $A$ и $B$ лежат по разные стороны от прямой $\ell$. Постройте на этой прямой точку $M$ так, чтобы прямая $\ell$ делила угол $AMB$ пополам.
    \item Внутри острого угла даны точки $M$ и $N$. Как из точки $M$ направить луч света, чтобы он, отразившись последовательно от сторон угла, попал в точку $N$?

\end{tasks}

\subsection{Площади}
% TODO: написать решения

\begin{tasks}
    \item\label{task:easy-varinion} Площадь прямоугольника равна 24. Найдите площадь четырехугольника с вершинами в серединах сторон прямоугольника.
    \item Средняя линия треугольника разбивает его на треугольник и четырехугольник. Какую часть составляет площадь полученного треугольника от площади исходного?
    \item Точка $M$ расположена на стороне $BC$ параллелограмма $ABCD$. Докажите, что площадь треугольника $AMD$ равна половине площади параллелограмма.
    \item Пусть $M$ --- точка на стороне $AB$ треугольника $ABC$, причем $AM : MB = m : n$. Докажите, что площадь треугольника $CAM$ относится к площади треугольника $CBM$ как $m : n$.
    \item Точки $M$ и $N$ --- соотвественно середины противоположных сторон $AB$ и $CD$ параллелограмма $ABCD$, площадь которого равна 1. Найдите площадь четырехугольника, образованного пересечениями прымях $AN$, $BN$, $CM$, $DM$.
    \item На сторонах $AB$ и $AC$ треугольника $ABC$, площадь которого равна 50, взяты соответственно точки $M$ и $K$ так, что $AM : MB = 1 : 5$, а $AK : KC = 3 : 2$. Найдите площадь треугольника $AMK$.
    \item Прямая, проведенная через вершину $C$ трапеции $ABCD$ параллельно диагонали $BD$, пересекает продолжение основания $AD$ в точке $M$. Докажите, что треугольник $ACM$ равновелик трапеции $ABCD$.
    \item Докажите, что медианы треугольника делят его на шесть равновеликих частей.
    \item Медианы $BM$ и $CN$ треугольника $ABC$ пересекаются в точке $K$. Докажите, что четырехугольник $AMKN$ равновелик треугольнику $BKC$.
    \item Точка внутри параллелограмма соединена со всеми его вершинами. Докажите, что суммы площадей треугольников, прилежащих к противоположным сторонам параллелограмма, равны между собой.
    \item Середины сторон выпуклого четырехугольника последовательно соединены отрезками. Докажите, что площадь полученного четырехугольника вдвое меньше площади исходного.\footnote{Привет \cref{task:easy-varinion}!}
    \item Отрезки, соединяющие середины противоположных сторон выпуклого четырехугольника, взаимно перпендикулярны и равны 2 и 7. Найдите площадь четырехугольника.
        \moditem{*} Докажите, что сумма расстояний от произвольной точки внутри равностороннего треугольника до его сторон всегда одна и та же.
    \item Докажите, что площадь треугольника равна произведению полупериметра треугольника и радиуса вписанной окружности.
    \item Дан треугольник $ABC$. Найдите геометрическое место таких точек $M$, для которых:
        \begin{tasks}
        \item треугольники $AMB$ и $ABC$ равновелики;
        \item треугольники $AMB$ и $AMC$ равновелики;
        \item треугольники $AMB$, $AMC$ и $BMC$ равновелики.
        \end{tasks}
    \item Окружность касается стороны треугольника, равной $a$, и продолжения двух других сторон. Докажите, что радиус окружности равен площади треугольника, деленной на разность между полупериметром и стороной $a$.
    \item Боковая сторона $AB$ и основание $BC$ трапеции $ABCD$ вдвое меньше ее основания $AD$. Найдите площадь трапеции, если $AC = a$, $CD = b$.
    \item Из середины каждой стороны остроугольного треугольника опущены перпендикуляры на две другие стороны. Докажите, что площадь ограниченного ими шестиугольника равна половине площади треугольника.

        % тут если будет слишко легко --- дать просто задачи 3 уровня из гордина. 
\end{tasks}

\subsection{Счёт углов-II}
\begin{enumerate}
    \item В треугольнике $ABC$ проведены высоты $BB_1$ и $CC_1$, а также отмечена точка $M$ -- середина стороны $BC$. Точка $H$ -- его ортоцентр, а точка $P$ -- пересечения луча \texttt{(!)} $MH$ с окружностью $(ABC)$. Докажите, что точки $P, A, B_1, C_1$  концикличны. 

    \solution{
        Отметим вторую точку пересечения $Q$ окружности $(ABC)$ с прямой $MH$. Тогда по \cref{cor:diametr} AQ -- диаметр, а значит $\angle APQ = 90^\circ$. Тогда $P$, $A$, $C_1$, $B_1$, $H$ концикличны, т.к. лежат на окружности с диаметром $AH$.
    }

    \item Во вписанном четырехугольнике $ABCD$ точка $P$ -- точка пересечения диагоналей $AC$ и $BD$. Точка $O$ -- центр окружности $(ABP)$. Докажите, что $OP \perp CD$. 

    \solution{
        Т.к. $ABCD$ -- вписанный, то $\triangle BAP \sim \triangle CPD$ (по двум углам). Тогда если $O_1$ -- центр окружности $(CPD)$, то $\angle APO = \angle DPO_1$.
        
        По \cref{th:OHisogonal} в треугольнике $CPD$, если $H_1$ -- его ортоцентр, $\angle DPO_1 = \angle CPH_1$. Тогда точки $O$, $P$, $H$ -- коллинеарны, т.к. $\angle CPH = \angle APO$ (вертикальные). А значит $OP \equiv PH \perp CD$. 
    }
    
    \item \named{Муниципальный этап ВСОШ (Москва), 2020, 9.4}{Пусть точки $B$ и $C$ лежат на по\-лу\-окруж\-но\-сти с диаметром $AD$. Точка $M$ -- середина отрезка $BC$. Точка $N$ такова, что точка $M$ -- середина отрезка $AN$, докажите что $BC \perp ND$}. 

    \solution{
        $ABNC$ -- параллелограмм. Тогда раз $AD$ -- диаметр, то $AB \perp BD$ и $AC \perp CD$. Но $AB \parallel CN$ и $AC \parallel BN$. Тогда $BD \perp CN$ и $CD \perp BN$. Значит $C$ -- ортоцентр треугольника $BND$, а значит $BC \perp ND$.
    }

    \item В треугольнике $ABC$ проведена высота $AD$ и отмечен центр описанной окружности -- $O$. Пусть точки $E$ и $F$ -- проекции точек $B$ и $C$ на прямую $AO$. $N$ -- точка пересечения прямых $AC$ и $DE$, а $M$ -- точка пересечения прямых $AB$ и $DF$. Докажите, что точки $A, D, N, M$ концикличны.

    \solution{
        Пусть точка $A'$ -- диаметрально противоположна $A$. Тогда $ACA' = \angle ABA' = 90^\circ$, отсюда $\angle CA'A = \angle ACF$ и $\angle BA'A = \angle ABE$. Т.к. $ABDE$ и $ADFC$ -- вписанные и по \cref{lem:concycle} $\angle ABE = \angle ADN$ и $\angle ACF = \angle ADM$. Тогда $\angle NDM = \angle BA'C$, а значит $ADNM$ -- вписанный, раз $ABA'C$ был вписанным. 
    }
    
    \item \named{Baltic Way, 2019, problem 12}{Let $ABC$ be a triangle and $H$ its orthocenter. Let $D$ be a point lying on the segment $AC$ and let $E$ be the point on the line $BC$ such that $BC \perp DE$. Prove that $EH \perp BD$ if and only if $BD$ bisects $AE$}. 

    \solution{
        Докажем в одну сторону, что если $BD$ разделила $AE$ пополам, то $EH \perp BD$. Пусть $X$ -- точка пересечения $AH$ и $DE$ Тогда раз $AH\equiv AX\perp BC \land DE \perp BC \Rightarrow AH \parallel DE$ и $BD\equiv XB$ делит $AE$ пополам, то значит $AXED$ -- параллелограмм, отсюда $XE \parallel AD$.  А раз $XE \parallel AD \land AD \perp BH$, значит $X$ -- ортоцентр треугольника $BHE$, а значит $BX\equiv BD\perp EH$.  
    }
\end{enumerate}

\begin{tasks}
    \item Докажите, что точка, симметричная точке пересечения высот (ортоцентру) треугольника относительно стороны, лежит на описанной окружности этого треугольника.

    \item Пусть точка $O$ --- центр описанной окружности треугольника $ABC$, $AH$ --- высота. Докажите, что $\angle BAH = \angle OAC$.
        
    \item Пусть $AA_1$ и $BB_1$ --- высоты остроугольного треугльника $ABC$, а точка $O$ --- центр его описанной окружности. Докажите, что $CO \perp A_1B_1$.

    \item В треугольнике $ABC$ проведены высоты $BB_1$ и $CC_1$, а также отмечена точка $M$ --- середина стороны $BC$. Точка $H$ --- его ортоцентр, а точка $P$ --- пересечения луча \texttt{(!)} $MH$ с окружностью $(ABC)$. Докажите, что точки $P, A, B_1, C_1$  концикличны. 

    \item Во вписанном четырехугольнике $ABCD$ точка $P$ --- точка пересечения диагоналей $AC$ и $BD$. Точка $O$ --- центр окружности $(ABP)$. Докажите, что $OP \perp CD$. 

    \item \named{Муниципальный этап ВСОШ (Москва), 2020, 9.4}{Пусть точки $B$ и $C$ лежат на по\-лу\-окруж\-но\-сти с диаметром $AD$. Точка $M$ --- середина отрезка $BC$. Точка $N$ такова, что точка $M$ --- середина отрезка $AN$, докажите что $BC \perp ND$}. 

    \item В треугольнике $ABC$ проведена высота $AD$ и отмечен центр описанной окружности --- $O$. Пусть точки $E$ и $F$ --- проекции точек $B$ и $C$ на прямую $AO$. $N$ --- точка пересечения прямых $AC$ и $DE$, а $M$ --- точка пересечения прямых $AB$ и $DF$. Докажите, что точки $A, D, N, M$ концикличны.

    \item Окружность $S_2$ проходит через центр $O$ окружности $S_1$ и пересекает ее в точках $A$ и $B$. Через точку A проведена касательная к окружности $S_2$; $D$ --- вторая точка пересечения этой касательной с окружностью $S_1$. Докажите, что $AD = AB$.

    \item \named{Baltic Way, 2019, problem 12}{Let $ABC$ be a triangle and $H$ its orthocenter. Let $D$ be a point lying on the segment $AC$ and let $E$ be the point on the line $BC$ such that $BC \perp DE$. Prove that $EH \perp BD$ if and only if $BD$ bisects $AE$}. 

    \item \named{Лемма Архимеда}{Две окружности касаются внутренним образом в точке $M$. Пусть $AB$ --- хорда большей окружности, касающаяся меньшей окружности в точке $T$. Докажите, что $MT$ ---  биссектриса угла $AMB$.}

    \item В трапеции $ABCD$ с основаниями $AB$ $CD$ выполнено равенство  $AB = BD+CD$. Пусть $𝐸$ --- середина $𝐴𝐶$. Докажите, что $\angle BED = 90^\circ$.

    \item В параллелограмме $ABCD$ диагональ $AC$ больше диагонали $BD$. Точка $M$ на диагонали $AC$ такова, что около четырехугольника $BCDM$ можно описать окружность. Докажите, что $BD$ --- общая касательная окружностей, описанных около треугольников $ABM$ и $ADM$.

    \item \named{Прямая Симсона}{Докажите, что основания перпендикуляров, опущенных из произвольной точки описанной окружности на стороны треугольника (или их продолжения), лежат на одной прямой.}

    \item Пусть $H$ --- ортоцентр остроугольного треугольника $ABC$ Серединный перпендикуляр $\ell$ к стороне $AC$ пересекает прямые $AH$, $CH$ в точках $K$ и $L$ соответственно. Докажите, что ортоцентр треугольника  лежит на прямой, содержащей одну из медиан треугольника $ABC$.

    \end{tasks}

\subsection{Подобие}    % TODO: взять из параграфа 2.9 Гордина
\input{tasks/simirality.tex}
\subsection{Степень точки и радикальная ось}    %TODO: взять задач из 3.1 Гордина
\begin{tasks}
    \item Диагонали $AC$ и $BD$ вписанного в окружность четырехугольника $ABCD$ взаимно перпендикулярны и пересекаются в точке $M$. Известно, что $AM = 3$, $BM = 4$ и $CM = 6$. Найдите $CD$.
    \item Через точку $M$ проведены две прямые. Одна из них касается некоторой окружности в точке $A$, а вторая пересекает эту окружность в точках $B$ и $C$, причем $BC = 7$ и $BM = 9$. Найдите $AM$.
    \item Дана точка $P$, удаленная на расстояние, равное $7$, от центра окружности, радиус которой равен $11$. Через точку $P$ проведена хорда, равная $18$. Найдите отрезки, на которые делится хорда точкой $P$.
    \item Точка $M$ лежит внутри окружности радиуса $R$ и удалена от центра на расстояние $d$. Докажите, что для любой хорды $AB$ этой окружности, проходящей через точку $M$, произведение $AM \cdot BM$ одно и то же. Чему оно равно?
    \item В квадрат $ABCD$ со стороной $a$ вписана окружность,
которая касается стороны $CD$ в точке $E$. Найдите хорду, соединяющую точки, в которых окружность пересекается с прямой $AE$.
    \item Диагональ $AC$ вписанного в окружность четырехугольника $ABCD$ является биссектрисой угла $BAD$. Докажите, что прямая $BD$ отсекает от треугольника $ABC$ подобный ему треугольник.
    \item Две окружности пересекаются в точках $A$ и $B$. Проведены хорды $AC$ и $AD$ этих окружностей так, что хорда одной окружности касается другой окружности. Найдите $AB$, если $CB = a$, $DB = b$.
    \item Докажите, что прямая, проходящая через точки пересечения двух окружностей, делит пополам общую касательную к ним.
    \item Продолжение медианы треугольника $ABC$, проведенной из вершины $A$, пересекает описанную окружность в точке $D$. Найдите $BC$, если $AC = DC = 1$.
    \item Сторона $AD$ квадрата $ABCD$ равна $1$ и является хордой некоторой окружности, причем остальные стороны квадрата лежат вне этой окружности. Касательная $BK$, проведенная из вершины $B$ к этой же окружности, равна $2$. Найдите диаметр окружности.
    \item Точка $B$ расположена между точками $A$ и $C$. На отрезках $AB$ и $AC$ как на диаметрах построены окружности. Прямая, перпендикулярная $AC$ и проходящая через точку $B$, пересекает большую окружность в точке $D$. Прямая, проходящая через
точку $C$, касается меньшей окружности в точке $K$. Докажите, что $CD = CK$.
    \item Постройте окружность, проходящую через две данные точки и касающуюся данной прямой.
    \item Докажите, что квадрат биссектрисы треугольника равен произведению сторон, ее заключающих, без произведения отрезков третьей стороны, на которые она разделена биссектрисой.
\end{tasks}

\begin{enumerate}
    \item Докажите, что высоты треугольника конкурентны. \texttt{0\_0} 

        \solution{
            Пусть $H_a$, $H_b$, $H_c$ -- основания высот треугольника $ABC$ из вершин $A$, $B$ и $C$ соответственно.

            Четырехугольники $ABH_aH_b$, $ACH_aH_c$ и $BCH_bH_c$ -- вписанные. По \cref{th:radcenter} прямые $AH_a$, $BH_b$, $CH_c$ конкурентны.
         }

    \item Окружность делит каждую из сторон треугольника на три равные части. Докажите, что этот треугольник -- равносторонний. 

        \solution{
            Пусть окружность высекает на сторонах $AB$, $AC$ и $BC$ треугольника $ABC$ отрезки $CC_1, BB_1, AA_1$. 

            \begin{equation}
                \begin{cases}
                    AB_2 = B_1B_2 = B_1C &= b \\
                    AC_1 = C_1C_2 = BC_2 &= c \\
                    BA_1 = A_1A_2 = A_2C &= a
                \end{cases}
            \end{equation}

            Т.к. $A_1A_2B_1B_2$ -- вписанный, то 
            \begin{equation}\label{eq:34.2}
                \begin{split}
                    \pow(C, (A_1A_2B_1B_2)) = CA_1 \cdot CA_2 = CB_1 \cdot CB_2 \implies\\
                    \implies c \cdot 2c = b \cdot 2b \implies c = b \implies AC = BC.
                \end{split}
            \end{equation}

            Т.к. $A_1A_2C_1C_2$ -- вписанный, то
            \begin{equation}\label{eq:34.3}
                \begin{split}
                    \pow(B, (A_1A_2C_1C_2)) = BA_1 \cdot BA_2 = BA_1 \cdot BA_2 \implies\\
                    \implies a \cdot 2a = c \cdot 2c \implies a = c \implies AB = AC.
                \end{split}
            \end{equation}

            Из уравнений \labelcref{eq:34.2,eq:34.3} следует что \(
            AB = AC = BC
            \).
        }

    \item\label{th:sekcircles} Окружности ${\color{red}{\psi}}$ и ${\color{blue}{\omega}}$ вписаны в вертикальный угол $\angle {\color{red}{n}}{\color{blue}{m}}$, ${\color{red}{\psi}}$ касается прямой ${\color{red}{n}}$ в точке ${\color{red}{N}}$, а ${\color{blue}{\omega}}$ касается прямой ${\color{blue}{m}}$ в точке ${\color{blue}{M}}$. Докажите, что ${\color{red}{\psi}}$ и ${\color{blue}{\omega}}$ высекают на ${\color{red}{N}}{\color{blue}{M}}$ равные отрезки. 

        \solution{
            Пусть окружность $\psi$ касается прямой $m$ в точке $Q$, а $\omega$ касается $n$ в точке $P$. Точка $R$ -- вторая точка пересечения прямой $MN$ с $\psi$. Точка $T$ -- вторая точка пересечения прямой $MN$ c $\omega$.

            По \cref{cor:tangent_and_sector}
            \begin{equation}
                \begin{aligned}
                    &\left\{\begin{aligned}
                        &\pow(M, \psi) = MN \cdot MR = MQ^2\\
                        &\pow(N, \omega) = NM \cdot NT = NP^2 \\
                        &MQ = NP, \quad \text{symmetry} 
                    \end{aligned}\right| \implies MN \cdot MR = NM \cdot NT\implies\\
                    &\implies MR = NT \implies MR - MN = NT - MN \implies NR = MT.
                \end{aligned}
            \end{equation}
            
        }
    \item \named{ММО, 2013, 11.3}{Четырёхугольник $ABCD$ такой, что $AB = BC$ и $AD = DC$. Точки $K$, $L$ и $M$ -- середины отрезков $AB$, $CD$ и $AC$ соответственно. Перпендикуляр, проведённый из точки $A$ к прямой $BC$, пересекается с перпендикуляром, проведённым из точки $C$ к прямой $AD$, в точке $T$. Докажите, что прямые $KL \perp TM$.} 
        
        \solution{
            Пусть точка $P$ -- основание перпендикуляра из точки $A$ на прямую $BC$, а точка $Q$ -- основание перпендикуляра из точки на прямую $AD$. 

            Т.к. $AB = BC$ и $AD=DC$, то $AC \perp BD$ и $AC \cap BD = M$.
            Тогда четырехугольники $APBM, BCDQ, APCQ$ -- вписанные, с центрами $K, L, M$ соответственно.

            По \cref{th:radcenter}
            \begin{equation}\label{eq:36.1}
                \begin{aligned}
                    &\left\{\begin{aligned}
                            AP &= \radaxis\left((AB), (AC)\right) \\
                            CQ &= \radaxis\left( (CD), (AC) \right) 
                        \end{aligned}\right| \implies \\
                    &\implies AP \cap CQ = T = \radcenter((AB), (AC), (CD)).
                \end{aligned}
            \end{equation} 
            
            По \cref{eq:36.1} 
            \begin{equation}
                M \in (AB) \cap (CD) \implies M \in \radaxis\left( (AB), (CD) \right) \implies KL \perp TM.
            \end{equation} 
        }
    \item Точка $D$ -- основание биссектрисы из точки $A$ треугольника $ABC$. Окружность $(ABD)$ повторно пересекает прямую $AC$ в точке $E$, а окружность $(ACD)$ повторно пересекает прямую $BC$ в точке $F$. Докажите, что $BF = CE$. 

        \solution{
            \begin{theorem}[Теорема о биссектрисе]\label{th:angle_bisector}
                В треугольнике $ABC$ провели биссектрису $AD$, тогда \[
                \frac{AB}{AC} = \frac{BD}{DC}
                .\] 
            \end{theorem}
            По \cref{th:superpow}

            \begin{equation}\label{eq:37.1}
                \left\{\begin{aligned}
                    \pow(B, (ADC)) = BF \cdot BA &= BD \cdot BC \\
                    \pow(C, (ADB)) = CE \cdot CA &= CD \cdot CB \\
                \end{aligned}\right| \implies \frac{BF \cdot BA}{BD} = \frac{CE \cdot CA}{CD} 
            .\end{equation} 

            По \cref{eq:37.1,th:angle_bisector}
            \begin{equation}
                \frac{BF}{CE} = \frac{BD \cdot CA}{BA \cdot CD} = \frac{BD}{CD} \cdot \frac{CA}{BA} = 1
            .\end{equation} 
        }

    \item Окружность $\omega$ проходит через вершины $A$ и $D$ равнобокой трапеции $ABCD$ и пересекает диагональ $BD$ и боковую сторону $CD$ в точках $P$ и $Q$ соответственно. Точки $P'$ и $Q'$ симметричны точкам $P$ и $Q$ относительно середин отрезков $BD$ и $CD$ соответственно. Докажите, что $B$, $C$, $P'$ и $Q'$ концикличны. 

        \solution{
            По обратной \cref{th:superpow}
            \begin{equation}\label{eq:38.1}
                \pow(C, \Omega) = DQ' \cdot DC = DP' \cdot DB.
            \end{equation}
            
            Если $P'$ и $Q'$ симметричны относительно середин отрезков $BD$ и $CD$, то $DP' = BP$ и $CQ = DQ'$. Тогда \cref{eq:37.1} преобразовывается в 
            \begin{equation}\label{eq:38.2}
                \underbrace{CQ \cdot CD}_{\pow(C, \omega)} = \underbrace{BP \cdot BD}_{\pow(B, \omega)}
            \end{equation}

            Уравнение \labelcref{eq:38.2} верно, т.к. $\omega, B, C$ -- все эти объекты симметричны относительно серединного перпендикуляра к $AD$.
        }

    \item \named{JBMO Shortlist, 2022, G6}{Пусть $\Omega$ -- описанная окружность треугольника $ABC$. Взяты точки $P$ и $Q$, так что $P$ равноудалена от $A$ и $B$, а $Q$ равноудалена от $A$ и $C$ и углы $PBC$ и $QCB$ равны. Докажите, что касательная к $\Omega$ в точке $A$, прямая $PQ$ и $BC$ пересекаются в одной точке.} 

        \solution{
            Пусть $\ell$ -- касательная в точке $B$ к окружности $(ABC)$.

            По \cref{cor:tangentangle} существует окружность $\omega$, которая касается прямой $AP$ в точке $A$, а прямой $BQ$ в точке $B$.
            \begin{equation}
                \left\{\begin{aligned}
                    AP^2 &= BP^2 \\
                    CQ^2 &= BQ^2                 
                \end{aligned}\right| \implies PQ = \radaxis((B), \omega).
            \end{equation}
            \begin{equation}
               \left\{\begin{aligned}
                    BC &= \radaxis(\omega, (ABC)) \\ 
                    PQ &= \radaxis((B), \omega)\\
                    \ell &= \radaxis((B), (ABC))
            \end{aligned}\right| \implies BC \cap PQ \cap \ell \neq \varnothing.
            \end{equation}
        }

    \item Вневписанные окружности $\omega_b$ и $\omega_c$ треугольника $ABC$ касаются сторон $AC$ и $AB$ соответственно в точках $E$ и $F$. Прямая $EF$ повторно пересекает окружности $\omega_b$ и $\omega_c$ в точках $X$ и $Y$ соответственно. Касательные в точках $X$ и $Y$ проведенные к окружностям $\omega_b$ и $\omega_c$ пересекают прямые $AC$ и $AB$ в точках $K$ и $L$ соответственно. Докажите, что середина отрезка $KL$ равноудалена от точек $E$ и $F$.

        \solution{
            По \cref{th:sekcircles} $EX = FY$.

            Пусть  $K', L'$ -- середины отрезков $EX, YF$ соответственно. Тогда $YL' = L'F = EK' = K'X$, $LL' \perp EF$ и $KK' \perp EF$. Тогда и середина $KL$ проецируется в середину $XY$, что эквивалентно середине $EF$.
        }

    \item \begin{enumerate}
            \item\label{lem:Hinradicalaxis}Пусть $C_1$ и $B_1$  -- точки на сторонах $AB$ и $AC$ треугольника $ABC$ соответственно. Докажите что, радикальная ось окружностей, построенных на $BB_1$ и $CC_1$ как на диаметре, проходит через ортоцентр треугольника $ABC$. 

            \solution{
                Пусть окружность $(BB_1)$ пересекает сторону $AC$ в точке $P$, а окружность $(CC_1)$ пересекает сторону $AB$ в точке $Q$.
                Тогда $BQ, CP$ -- высоты треугольника $ABC$, тогда $BQ \cap CP = H$ -- ортоцентр.

                Построим окружность $(BC) \subset \{P, Q\}$. 

                \begin{equation}
                    \begin{aligned}
                        &\left\{\begin{aligned}
                                BQ &= \radaxis((BC), (BB_1)) \\
                                CP &= \radaxis((BC), (CC_1)) \\
                            \end{aligned}\right| \implies \\ 
                        &\implies H = \radcenter((BC), (BB_1), (CC_1)) \implies \\
                        &\implies H \in \radaxis((BB_1), (CC_1)).
                    \end{aligned}
                \end{equation}
            }

        \item \label{def:Ober's_axis}\named{Ось Обера}{Докажите, что четыре ортоцентра четырёх треугольников, образованных четырьмя попарно пересекающимися прямыми, никакие три из которых не проходят через одну точку\footnote{Такие прямые образуют фигуру, называемую полным четырёхсторонником.}, коллинеарны.}

            \solution{
                Пусть треугольник $ABC$ пересекает прямая $\ell$, которая пересекает стороны $AB,AC,BC$ в точках $C_1,B_1,A_1$ соответственно.
                Через $H_{ABC}, H_{A_1B_1C}, H_{A_1BC_1}, H_{AB_1C_1}$ будем обозначать ортоцентры соотетствующих треугольников.

                Построим на $AA_1, BB_1, CC_1$ окружности как на диаметрах.
                Тогда по \cref{lem:Hinradicalaxis} для треугольника $ABC$
                 \begin{equation}
                    \left\{\begin{aligned}
                        H_{ABC} &\in \radaxis((AA_1), (BB_1)) \\
                        H_{ABC} &\in \radaxis((AA_1), (CC_1)) \\
                        H_{ABC} &\in \radaxis((BB_1), (CC_1))
                    \end{aligned}\right.
                \end{equation} 

                Аналогичные утверждения можно произвести для других ортоцентров, таким образом получается, что каждый ортоцентр лежит на каждой радикальной оси каждой пары окружности.
                Т.к. ортоцентры различны, то радикальные оси не могут пересекаться в одной точке, а значит радикальные оси совпадают. И каждый ортоцентр лежит на этой общей радикальной оси.
            }

        \item \named{Теорема Гаусса-Боденмиллера}{Докажите, что прямая Гаусса\footnote{Прямой Гаусса полного четырёхсторонника называется прямая, проходящая через середины трех его диагоналей.} перпендикулярна оси Обера.}

            \solution{
                По \cref{th:radaxis,def:Ober's_axis} \emph{Ось Обера} будет перпендикулярна линии центров данных окружностей. А линия центров данных окружностей и есть \emph{прямая Гаусса}, т.к. центрами окружностей являются центры диагоналей четырехсторонника.
            }

    \end{enumerate}
    \item Чевианы $AD$, $BE$ и $CF$ треугольника $ABC$ конкурентны. Прямая $EF$ пересекает окружность $(ABC)$ в точках $P$ и $Q$. Докажите, что $P$, $Q$, $D$ и середина отрезка $BC$ концикличны.

        \solution{
            \begin{theorem}[Теорема Чевы]\label{th:chev}
                Чевианы $AA_1, BB_1, CC_1$ треугольника $ABC$ конкурентны тогда и только тогда, когда 
                \[
                \frac{AC_1}{C_1B}\cdot \frac{BA_1}{A_1C}\cdot \frac{CB_1}{B_1A} = 1
                .\] 
            \end{theorem}

            \begin{theorem}[Теорема Менелая]\label{th:menel}
                Точки $A_1, B_1, C_1$ на прямых $BC, AC, AB$ соответственно коллинеарны тогда и только тогда, когда 
                \[
                \frac{AC_1}{C_1B}\cdot \frac{BA_1}{A_1C}\cdot \frac{CB_1}{B_1A} = 1
                .\] 
            \end{theorem}

            Пусть прямая $PQ$ пересекает прямую $BC$ в точке $T$, а точка $M$ -- середина $BC$.

            \begin{equation}\label{eq:42.1}
                \pow(T, (ABC)) = TP \cdot TQ = TB \cdot TC.
            \end{equation}

            Чтобы искомая окружность $\omega$ существовало должно выполняться

            \begin{equation}\label{eq:42.2}
                \pow(T, \omega) = TD \cdot TM = \underbrace{TP \cdot TQ = TB \cdot TC}_{\text{по \cref{eq:42.1}}}.
            \end{equation}

            Также по \cref{th:chev,th:menel}

            \begin{equation}\label{eq:42.3}
                \frac{BT}{CT} \underset{\text{по \cref{th:menel}}}{=} \frac{BF}{FA}\cdot \frac{AE}{EC} \underset{\text{по \cref{th:chev}}}{=} \frac{BD}{DC}.
            \end{equation}

            Заметим что в уравнениях \labelcref{eq:42.2,eq:42.3} остались только точки на прямой $BC$. Такую задачу можно решить координатным способом, за начало координат приняв $T$. 
            \begin{equation}\label{eq:42.4}
                \begin{aligned}
                    \frac{TB}{TC} &= \frac{BD}{DC} = \frac{TD-TB}{TC-TD} \Longleftrightarrow \\ 
                    \Longleftrightarrow TB(TC-TD) &= TC(TD-TB) \\
                    TB\cdot TC - TB \cdot TD &= TC \cdot TD - TB\cdot TC \\
                    2 TB\cdot TC &= TD \left( TC + TB \right) \\
                    TB \cdot TC &= TD \cdot \frac{TC+TB}{2} = TD \cdot TM. \\
                \end{aligned} 
            \end{equation} 

            Хочется еще отметить, что из уравнения \labelcref{eq:42.4} следует, что 
            \[
                TD = \frac{2TB\cdot TC}{TB + TC} = \frac{2}{\frac{1}{TB}+\frac{1}{TC}}
            .\] 
            
            Поэтому четверка точек  $\left( T, B, D, C \right) $ называется \emph{гармонической}.
        }
    %\item \named{Устная олимпиада по геометрии, 2014, 10-11.4}{Медианы $AM_a$, $BM_b$ и $CM_c$ остроугольного треугольника $ABC$ пересекаются в точке $G$, а высоты $AH_a$, $BH_b$ и $CH_c$ -- в точке $H$. Касательная к окружности девяти точек треугольника $ABC$ а в точке $H_c$ пересекает прямую $M_aM_b$ в точке $C'$. Точки $A'$ и $B'$ определяются аналогично. Докажите, что $A'$, $B'$ и $C'$ лежат на одной прямой, перпендикулярной прямой $GH$.}

    \item В треугольнике $ABC$ проведены высоты $AD$, $BE$, $CF$. Прямые $DE$, $EF$ и $DF$ пересекаются прямые $AB$, $BC$ и $AC$. В точках $C_1$, $B_1$, $A_1$ соответственно. Докажите, что точки $A_1$, $B_1$, $C_1$ лежат на прямой\footnote{Такая прямая называется трилинейной полярой ортоцентра, или ортоцентрической осью, или центральной линией центра описанной окружности.} перпендикулярной прямой Эйлера треугольник $ABC$.

        \solution{
            По теореме об окружности Эйлера Точки $D, E, F$ лежат на окружности Эйлера $\omega_9$ треугольника $ABC$. А $\Omega$ -- описанная окружность этого треугольника.

            Каждый из четырехугольников $ABDE, BCEF, CAFD$ является вписанным.
            \begin{equation}
                \left\{\begin{aligned}
                    \underbrace{A_1B \cdot A_1C}_{\pow(A_1, \omega_9)} &= \underbrace{A_1F \cdot A_1E}_{\pow(A_1, \Omega)} \\ 
                    \underbrace{B_1C \cdot B_1A}_{\pow(B_1, \omega_9)} &= \underbrace{B_1D \cdot B_1F}_{\pow(B_1, \Omega)} \\ 
                    \underbrace{C_1A \cdot C_1B}_{\pow(C_1, \omega_9)} &= \underbrace{C_1E \cdot C_1D}_{\pow(C_1, \Omega)}
                \end{aligned}\right| \implies \{A_1, B_1, C_1\} \in \radaxis(\omega_9, \Omega)
            .\end{equation} 
        }

    %\item \named{ММО, 2013, 10.6}{Пусть $I$ -- инцентр неравнобедренного треугольника $ABC$. $A_1$ -- середина дуги $BC$ описанной окружности треугольника $ABC$, не содержащей точки $A$, а $A_2$ -- середина дуги $BAC$. Перпендикуляр, опущенный из точки $A_1$ на прямую $A_2I$, пересекает прямую $BC$ в точке $A'$. Аналогично определяются точки $B'$ и $C'$. \begin{enumerate}
        %\item Докажите, что точки $A'$, $B'$, $C'$ коллинеарны.
        %\item Докажите, что эта прямая перпендикулярна прямой $OI$%\footnote{Можно рассматривать степень точки относительно вырожденной окружности.}, где $O$ -- центр описанной окружности треугольника $ABC$.
    %\end{enumerate}}
\end{enumerate}


\newpage
\markboth{Контрольная работа}{}
\addcontentsline{toc}{section}{Контрольная работа}
\section*{Контрольная работа}
% \input{tasks/exam.tex}

\end{document}
