\begin{figure}[ht]
    \centering
    \begin{asy}
        size(6.3cm);
        point A = (-5, 0); point B = (0, 4);
        point C = (3, 0); point D = (0, -2);
        draw(A--B--C--D--A, linewidth(bp));

        draw(A--C, red); draw(B--D, red);

        perpendicularmark(line(A, C), line(B, D), blue+dashed, size=10, quarter=1);

        dot("$A$", A, W); dot("$B$", B, N); dot("$C$", C, E); dot("$D$", D, S);
        dot(intersectionpoint(line(B, D), line(A, C)));
    \end{asy}
    \hfill
    \begin{asy}
        size(6.3cm);
        point A = (-5, 0); point B = (0, 6);
        point C = (3.5, 0); point D = (0, 2.5);
        draw(A--B--C--D--A, linewidth(bp));

        point H = intersectionpoint(line(A, C), line(B, D)); 
        draw(A--C, red); draw(B--H, red);

        perpendicularmark(line(A, C), line(B, H), blue+dashed, size=10, quarter=2);
        
        dot("$A$", A, W); dot("$B$", B, N); dot("$C$", C, E); dot("$D$", D, dir(150)); dot(H);
    \end{asy}
    \caption{Ортодиагональные четырёхугольники (выпуклый и невыпуклый).}
\end{figure}

\begin{theorem}\label{th:diagonals}
    Диагонали $AC$ и $BD$ четырехугольника $ABCD$ (выпуклого или не выпуклого) перпендикулярны тогда и только тогда, когда $$AB^2 + CD^2 = BC^2 + AD^2.$$
\end{theorem}
