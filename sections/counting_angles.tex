Под этим названием скрывается, не побоюсь этого слова, самый (!) используемый метод в решении задач. В каждой он встречается в том или ином виде. Поэтому, если вы хотите решать задачи, вам нужно его знать. В основном ``считаются'' углы, связанные с окружностями, но бывает и что-то другое.

Для примера, давайте дакажем, что высоты треугольника пересекаются в одной точке. Для этого вспомним <<вписанные углы>>
\begin{theorem}\label{th:otrhocenter}
    Высоты треугольника \emph{конкурентны\footnote{Пересекаются в одной точке.}}.
\end{theorem}
\begin{figure}[ht]
    \centering
    \begin{asy}
        size(9cm);
        triangle t = triangleabc(9, 10, 11); draw(t, linewidth(bp));

        point H_a = foot(t.VA); point H_b = foot(t.VB); point H_c = foot(t.VC);
        point H = orthocentercenter(t);

        draw(segment(t.VA, H_a), grey); draw(segment(t.VB, H_b), grey); draw(segment(t.VC, H_c), grey);

        perpendicularmark(t.AB, line(t.VC, H_c), dashed + blue, size=10);
        perpendicularmark(t.BC, line(t.VA, H_a), blue, size=10, quarter=1);
        perpendicularmark(t.AC, line(t.VB, H_b), blue, size=10, quarter=3);

        draw(circle(t.VC, H_a, H_b));
        draw(H_a--H_b, grey);
        
        markangle(n = 1, H_a, H, point(t.VC), radius=12, red);
        markangle(n = 1, H_a, H_b, point(t.VC), radius=14, red);

        clipdraw(circle(t.VA, H_a, H_b));

        //draw(arc(circle(t.VA, H_a, H_b), -5, 185));

        markangle(n = 1, H_a, point(t.VB), H_c, radius=16, red);

        draw(circle(H, H_a, H_c), dashed+red+bp*0.7);
    \end{asy}
    \caption{Высоты треугольника пересекаются в одной точке.}
    \label{fig:orthocenter}
\end{figure}

\noindent
\begin{minipage}{0.65\linewidth}
    \begin{lemma}\label{lem:concycle}
        Четырехугольник $ABCD$ является вписанным, если $\angle ABC$ равен смежному углу $\angle ADC$.
    \end{lemma}
\end{minipage}
\hspace{0.05\linewidth}
\begin{minipage}{0.3\linewidth}
    \begin{asy}
        size(3.5cm);
        point O = (0, 0);
        circle omega = circle(O, 1); draw(omega);

        point A = angpoint(omega, 210);
        point B = angpoint(omega, 100);
        point C = angpoint(omega, 15);
        point D = angpoint(omega, 320);

        point P = intersectionpoints(circle(B, 2), line(A, B))[0];
        
        draw(P--A--B--C--D--A);
        markangle(n = 1, line(B, A), line(A, D), radius=12, red+dashed);
        markangle(n = 1, B, C, D, radius=14, red);

        markangle(n = 2, D, A, B, radius=14, blue);
    \end{asy}
\end{minipage}

\begin{proposition}\label{prop:angle-between-tangent-and-chord}
    Пусть $AB$ -- хорда окружности, а $C$ -- точка касания касательной к окружности. Тогда угол между касательной и хордой равен вписанному углу, операющему на ту же дугу, что и хорда. То есть \[\angle ACB = \frac{\overset{\frown}{AB}}{2}.\]
\end{proposition}
\begin{proof}
    Пусть $O$ -- центр окружности. Тогда отрезки $OB$ и $ОС$ равны как радиусы. При том, угол $\angle BOC = 2\cdot \angle BAC$. Радиус $OC$ перпенидкулярен касательной в точке $C$. Значит угол между касательной и хордой равен:
    \[
        90^\circ - \frac{180^\circ - 2\cdot\angle BAC}{2} = \angle BAC.
    \]
\end{proof}

\begin{figure}
    \centering
    \begin{asy}
        size(6cm, 5cm);
        point O = (0, 0);
        circle omega = circle(O, 1); draw(omega);

        point A = angpoint(omega, 210);
        point B = angpoint(omega, 70);
        point C = angpoint(omega, 5);

        line t = tangent(omega, C); draw(t, red);

        draw(A--B);
        draw(C--A);
        draw(C--B, blue);

        markangle(n = 1, t, line(B, C), radius=20, red);
        markangle(n = 2, C, A, B, radius=20, red);

        dot("$A$", A, dir(210));
        dot("$B$", B, dir(90));
        dot("$C$", C, dir(5));
    \end{asy}
    \caption{Угол между касательной и хордой.}
    \label{fig:angle-between-tangent-and-chord}
\end{figure}
