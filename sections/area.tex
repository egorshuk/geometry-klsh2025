Давайте сейчас повторим то, чего мы уже знаем про площади различных многоугольников.

\begin{itemize}
    \item Равные многоулольники имеют равную площадь.
    \item Если многоугольник составлен из нескольких многоугольников, то его площадь равняется сумме площадей этих многоугольников.
    \item Площадь прямогоульника равняется прозведению двух его сторон.
    \item Площадь треугольника равняется половине произведения его основания на высоту.
    \item Площадь параллелограмма равна произведению его основания на высоту.
    \item Площадь ромба равна половине произведения его диагоналей
    \item Площадь трапеции равна произведению ее средней линии на высоту.
    \item Фигуры, имеющие равные площади, называются равновеликими. 
\end{itemize}

% TODO: нужно сделать картинки для всего-всего тут.

\begin{example}
    Докажите, что в треугольнике высоты обратно пропорциональна сторонам, к которым они проведены.
\end{example}

\begin{example}
    Медины $AA_1$ и $BB_1$ треугольника $ABC$ пересекаются в точке $M$. Найдите площадь треугольника $A_1MB_1$, если площадь треугольника $ABC$ равняется 1.
\end{example}

\begin{example}
    Докажите, что биссектриса треугольника делит его сторону на отрезки, пропорциональные двум другим сторонам.
\end{example}
