\begin{theorem}[Прямая Симсона]\label{th:simson's line}
    Проекции точки $P$ на прямые, содержащие стороны треугольника $ABC$, коллинеарны, тогда и только тогда, когда точка $P$ лежит на описанной окружности треугольника $ABC$.
\end{theorem}
\begin{figure}[ht]
    \centering
    \begin{asy}
        size(12cm, 9.8cm);
        triangle t = triangleAbc(70, 7, 11); drawline(t, gray); draw(t, linewidth(bp));

        circle omega = circle(t);
        draw(omega);

        point P = angpoint(omega, 45);
        point Q = (-1, 6);
        

        point P_a = intersectionpoint(pedal(t.BC, P), t.BC);
        point P_b = intersectionpoint(pedal(t.AC, P), t.AC);
        point P_c = intersectionpoint(pedal(t.AB, P), t.AB);

        point Q_a = intersectionpoint(pedal(t.BC, Q), t.BC);
        point Q_b = intersectionpoint(pedal(t.AC, Q), t.AC);
        point Q_c = intersectionpoint(pedal(t.AB, Q), t.AB);

        draw(segment(P, P_a), grey);
        draw(segment(P, P_b), grey);
        draw(segment(P, P_c), grey);

        draw(segment(Q, Q_a), grey);
        draw(segment(Q, Q_b), grey);
        draw(segment(Q, Q_c), grey);
        
        perpendicularmark(t.BC, line(P, P_a), blue, size=10, quarter=3);
        perpendicularmark(t.AC, line(P, P_b), blue, size=10, quarter=3);
        perpendicularmark(t.AB, line(P, P_c), blue, size=10, quarter=2);

        perpendicularmark(t.BC, line(Q, Q_a), blue, size=10, quarter=2);
        perpendicularmark(t.AC, line(Q, Q_b), blue, size=10, quarter=2);
        perpendicularmark(t.AB, line(Q, Q_c), blue, size=10, quarter=1);
        
        draw(line(P_a, P_c), red+dashed);
        dot(P_a, red+3); dot(P_b, red+3); dot(P_c, red+3); 
        dot(P, red+4);
        draw(Q_a--Q_b--Q_c--Q_a, red+dashed);
        dot(Q_a, red+3); dot(Q_b, red+3); dot(Q_c, red+3); 
        dot(Q, red+4);
        draw(box((-1.5, -5), (12, 9)), invisible);
        
        //clip(currentpicture, box((-1.5, -5), (12, 9.5)), invisible);
    \end{asy}
    \caption{Педальные треугольники двух точек. Прямая Симсона.}
\end{figure}
