\begin{lemma}\label{lem:con-or-kite}
    Если в четырехугольнике $ABCD$, $AC$ -- биссектриса угла $A$ и $BC = CD$, то этот четырехугольник является либо вписанным, либо дельтойдом\footnote{\emph{Дельтойдом (кайтом)} называется четырехугольик, у которого есть две пары равных смежных сторон}.
\end{lemma}

\begin{figure}[h]
    \centering
    \begin{asy}
        size(6cm, 5cm);
        triangle t = triangleabc(13, 9, 10);

        circle omega = circle(t); 
        point M = intersectionpoints(omega, bisector(t.VC))[0];

        draw(point(t.VA)--point(t.VC)--point(t.VB)--M--cycle, linewidth(bp));

        draw(point(t.VC)--M, blue);
        markangle(n = 1, point(t.VA), point(t.VC), M, radius=16, red);
        markangle(n = 1, M, point(t.VC), point(t.VB), radius=18, red);

        draw(point(t.VA)--M, StickIntervalMarker(1, 1, size=7));
        draw(point(t.VB)--M, StickIntervalMarker(1, 1, size=7));

        point H = intersectionpoint(t.BC, perpendicular(t.VA, line(t.VC, M)));
        draw(M--H, dashed+red, StickIntervalMarker(1, 1, size=7));
        dot(H);

        draw(omega, red+dashed);

        label("$A$", point(t.VC), N);
        label("$B$", point(t.VB), E);
        label("$C$", M, S);
        label("$D$", point(t.VA), W);
    \end{asy}
    \label{fig:con-or-kite}
    \caption{Случай вписанности и дельтойда.}
\end{figure}
