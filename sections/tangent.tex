Задачи на касающиеся объекты периодически попадются на олимпиадах. Могут касаться две окружности, а также прямая и окружность. Чтобы было удобнее обсуждать эту тему изучим еще одну новую технику -- счет в дугах.

\subsubsection{Счёт в дугах}

\begin{lemma}\label{lem:angle-between-chords}
    Две прямые $AB$ и $CD$ пересекаются внутри окружности в точке $P$, а точки $A$, $B$, $C$ и $D$ лежат на окружности. Тогда \[\angle APB = \frac{\overset{\frown}{AC} + \overset{\frown}{BD}}{2}.\]
    Здесь $AB$ и $CD$ -- это меньшие дуги окружности.
\end{lemma}

\begin{figure}[h]
    \centering
    \begin{asy}
        size(5cm);
        point O = (0, 0);
        circle omega = circle(O, 1); draw(omega);

        point A = angpoint(omega, 210);
        point C = angpoint(omega, 100);
        point B = angpoint(omega, 15);
        point D = angpoint(omega, 320);

        point P = intersectionpoint(line(A, B), line(C, D));


        draw(A--B, blue); draw(C--D, blue);
        draw(A--D, red+dashed);
        markangle(n = 3, C, P, A, radius=14, red);
        dot(P);
        markangle(n = 1, C, D, A, radius=14, red);
        markangle(n = 2, D, A, B, radius=14, red);
        dot("$A$", A, dir(210));
        dot("$B$", B, dir(15));
        dot("$C$", C, dir(100));
        dot("$D$", D, dir(320));
        dot("$P$", P, dir(50));
    \end{asy}
    \caption{Угол между хордами.}
    \label{fig:angle-between-chords}
\end{figure}

\begin{lemma}\label{lem:angle-between-secant}
    Две прямые $AB$ и $CD$ пересекаются вне окружности в точке $P$, а точки $A$, $B$, $C$ и $D$ лежат на окружности. Тогда \[\angle APB = \frac{\overset{\frown}{AC} - \overset{\frown}{BD}}{2}.\] Здесь $AB$ и $CD$ -- это меньшие дуги окружности. 
\end{lemma}
\begin{figure}[h]
    \centering
    \begin{asy}
        size(6cm, 5cm);
        point O = (0, 0);
        circle omega = circle(O, 1); draw(omega);

        point A = angpoint(omega, 210);
        point C = angpoint(omega, 90);
        point D = angpoint(omega, 5);
        point B = angpoint(omega, 320);

        point P = intersectionpoint(line(A, B), line(C, D));
        draw(A--P, blue); draw(C--P, blue);
        draw(A--D, red+dashed);
        markangle(n = 3, C, P, A, radius=14, red);
        dot(P);
        markangle(n = 1, C, D, A, radius=14, red);
        markangle(n = 2, B, A, D, radius=14, red);
        dot("$A$", A, dir(210));
        dot("$B$", B, dir(320));
        dot("$C$", C, dir(90));
        dot("$D$", D, dir(5));
        dot("$P$", P);
    \end{asy}
    \caption{Угол между секущими к окружности.}
    \label{fig:angle-between-secant}
\end{figure}

Эта техника и сама бывает стоящей полезной в несложных задачах. Сейчас мы с помощью нее и вычислим угол между касательной и хордой.

\begin{proposition}\label{prop:angle-between-tangent-and-chord}
    Пусть $AB$ -- хорда окружности, а $C$ -- точка касания касательной к окружности. Тогда угол между касательной и хордой равен вписанному углу, операющему на ту же дугу, что и хорда. То есть \[\angle ACB = \frac{\overset{\frown}{AB}}{2}.\]
\end{proposition}
\begin{proof}
    Пусть $P$ -- точка пересечения касательной с прямой $AB$. Тогда можем посчитать угол $\angle APC$ по \cref{lem:angle-between-secant}:
    \[
        \angle APC = \frac{\overset{\frown}{AC} - \overset{\frown}{BC}}{2}.
    \]

\end{proof}

\begin{figure}
    \centering
    \begin{asy}
        size(6cm, 5cm);
        point O = (0, 0);
        circle omega = circle(O, 1); draw(omega);

        point A = angpoint(omega, 210);
        point B = angpoint(omega, 70);
        point C = angpoint(omega, 5);

        line t = tangent(omega, C);
        point P = intersectionpoint(line(A, B), t);

        draw(t, red);
        draw(A--B);
        draw(C--A);
        draw(C--B, blue);
        draw(B--P, dashed);

        markangle(n = 1, t, line(B, C), radius=20, red);
        markangle(n = 2, C, A, B, radius=20, red);
        markangle(n = 3, A, P, C, radius=20, blue);

        dot("$A$", A, dir(210));
        dot("$B$", B, dir(90));
        dot("$C$", C, dir(5));
        dot("$P$", P, dir(40));
    \end{asy}
    \caption{Угол между касательной и хордой.}
    \label{fig:angle-between-tangent-and-chord}
\end{figure}
